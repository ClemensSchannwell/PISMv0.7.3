
\section{Practical usage}
\label{sec:practical-usage}

\subsection{Choosing variables to save at the end of the run}
\label{sec:output-size}

A PISM output file always contains variables that are necessary to restart it
plus some diagnostic quantities. It is possible, though, to make this list
bigger (or smaller) using the \intextoption{o\und size} command-line option.

Possible choices are ``\texttt{-o\und size small}'' (no diagnostic variables),
``\texttt{medium}'' (the default, adds variables listed in the \texttt{output\und
  medium} configuration variable to model state fields) and ``\texttt{big}'',
saving \emph{every} variable listed in Table \ref{tab:extra-vars}.

\subsection{Saving time series of scalar diagnostic quantities}\index{time-series}\index{PISM!saving time-series}
\label{sec:saving-time-series}
 It is also possible to save time-series of certain scalar diagnostic quantities using a combination of the options \intextoption{ts\und file}, \intextoption{ts\und times}, and \intextoption{ts\und vars}.  For example,
\begin{verbatim}
$ pismr -i foo.nc -y 1e4 -o output.nc -ts_file time-series.nc \
        -ts_times 0:1:1e4 -ts_vars ivol,iareag
\end{verbatim} %$ just to make the text editor happy
will run for 10000 years, saving total ice volume and grounded ice area to \texttt{time-series.nc} \emph{yearly}. See tables \ref{tab:time-series-opts} for the list of options and \ref{tab:time-series} for the full list of supported time-series.

Note that, similarly to the snapshot-saving code, this mechanism does not affect adaptive time-stepping.  Here, however, PISM will save exactly the number of time-series records requested, \emph{linearly interpolated onto requested times}.

PISM buffers time-series data and writes it at the end of the run or once 10000 values are stored, whichever comes first. Sending an \texttt{USR1} signal to a PISM process flushes these buffers, making it possible to monitor the run. (See section \ref{subsect:signal} for more about PISM's signal handling.)

Omitting the \intextoption{ts\und vars} option makes PISM save \emph{all} the variables listed in table \ref{tab:time-series}.  Because scalar time-series take very little storage space, compared to spatially-varying data, this is usually a reasonable choice.

If the file \verb|foo.nc|, specified by \verb|-ts_file foo.nc|, already exists then by default the existing file will be moved to \verb|foo.nc~| and the new time series will go into \verb|foo.nc|.  To append the time series onto the end of the existing file, use option \intextoption{ts\und append}.

\begin{table}[ht]
  \caption{Command-line options controlling saving scalar time-series}
  \centering
  \begin{tabular}{p{0.35\linewidth}p{0.55\linewidth}}\hline
    \textbf{Option} & \textbf{Description} \\
    \hline
    \intextoption{ts\und file} [file name] & Specifies the file to save to.\\
    \intextoption{ts\und times} [range or list] & Specifies times to save at either as a MATLAB-style range $a:\Delta t:b$ or a comma-separated list. \\
    \intextoption{ts\und vars} [list of variables] & Comma-separated list of variables, see table~\ref{tab:time-series}. Omitting this option is equivalent to listing all the possible variables.\\
    \intextoption{ts\und append} & Append time series to file if it already exists.  No effect if file does not yet exist. \\
    \hline
  \end{tabular}
 \label{tab:time-series-opts}
\end{table}

\begin{table}[ht]
  \caption{Scalar time-series supported by PISM}
  \centering
  \begin{tabular}{p{0.4\linewidth}p{0.1\linewidth}p{0.4\linewidth}}\hline
   \textbf{Variable name} & \textbf{Units} & \textbf{Description}\\
   \hline
    \texttt{ivol} &m$^{3}$ & total ice volume\\
    \texttt{iarea} & m$^{2}$ & total area covered by ice \\
    \texttt{iareag} & m$^{2}$ & area covered by grounded ice\\
    \texttt{iareaf} & m$^{2}$ & area covered by floating ice\\
    \texttt{dt} & years & mass-continuity time-step\\
    \texttt{divoldt} & m$^{3}\,$s$^{-1}$ & ice volume rate of change\\
    \texttt{dimassdt} & kg$^{3}\,$s$^{-1}$ & ice mass rate of change\\
    \texttt{total\und surface\und ice \und flux} & kg$^{3}\,$s$^{-1}$ & total surface ice flux \\
    \texttt{total\und basal\und ice\und flux} & kg$^{3}$\,s$^{-1}$ & total basal ice flux \\
    \texttt{total\und sub\und shelf\und ice\und flux} & kg$^{3}\,$s$^{-1}$  & total sub-ice-shelf ice flux \\
    \texttt{ienthalpy} & J & total ice enthalpy\\
    \hline
  \end{tabular}
 \label{tab:time-series}
\end{table}


\subsection{Saving time series of spatially-varying diagnostic quantities}\index{PISM!saving diagnostic quantities regularly}
\label{sec:saving-spat-vari}

Sometimes it is useful to have PISM save a handful of diagnostic quantities every 10 years (or even every year).  One can use snapshots (section \ref{sec:snapshots}), but doing so can easily fill your hard-drive because snapshots are complete (re-startable) model states.  Sometimes you want a \emph{subset} of model variables saved reasonably-frequently in an output file.

Use options \intextoption{extra\und file}, \intextoption{extra\und times}, and \intextoption{extra\und vars} for this.  For example,
\begin{verbatim}
$ pismr -i foo.nc -y 10000 -o output.nc -extra_file extras.nc \
        -extra_times 0:10:1e4 -extra_vars csurf,cbase
\end{verbatim}
will run for 10000 years, saving the magnitude of horizontal velocities at the ice surface and at the base of ice every 10 years. Times are specified using a comma-separated list or a MATLAB-style range. Please see table \ref{tab:extras} for all the options corresponding to this feature; table \ref{tab:extra-vars} lists all the variable choices. 

If the file \verb|foo.nc|, specified by \verb|-extra_file foo.nc|, already exists then by default the existing file will be moved to \verb|foo.nc~| and the new time series will go into \verb|foo.nc|.  To append the time series onto the end of the existing file, use option \intextoption{extra\und append}.

\begin{table}[ht]
  \caption{Command-line options controlling extra diagnostic output}
  \centering
  \begin{tabular}{p{0.35\linewidth}p{0.55\linewidth}}\hline
    \textbf{Option} & \textbf{Description}\\
    \hline
    \intextoption{extra\und file} [file name] & Specifies the file to save to; should be different from the output \intextoption{o} file.\\
    \intextoption{extra\und times} [range or list] & Specifies times to save at either as a MATLAB-style range $a:\Delta t:b$ or a comma-separated list.\\
    \intextoption{extra\und vars} [list of variables]& Comma-separated list of variables, see table~\ref{tab:extra-vars}\\
    \intextoption{extra\und split} & Save to separate files, similar to \intextoption{split\und snapshots}\\
    \intextoption{extra\und append} & Append variables to file if it already exists.  No effect if file does not yet exist, and no effect if \intextoption{extra\und split} is set. \\
    \hline
  \end{tabular}
 \label{tab:extras}
\end{table}

\begin{table}[ht]
  \caption{Diagnostic quantities}
  \centering
\small \hspace{-2mm}
  \begin{longtable}{p{0.15\linewidth}p{0.15\linewidth}p{0.6\linewidth}}\hline
    \textbf{Name} & \textbf{Units} & \textbf{Description}\\
    \hline
    \texttt{acab} &m/year & instantaneous ice-equivalent surface mass balance (accumulation/ablation) rate \\
    \texttt{age} & years & age of ice \\
    \texttt{artm} & K & time-dependent annual average ice temperature at ice surface but below firn processes\\
    \texttt{bfrict} & mW/m$^{2}$ & basal frictional heating from ice sliding (= till dissipation) \\
    \texttt{bheatflx} & W/m$^{2}$ & upward geothermal flux at bedrock surface \\
    \texttt{bmelt} & m/year & ice basal melt rate in ice thickness per time \\
    \texttt{bwat} & m & effective thickness of subglacial melt water \\
    \texttt{bwp} & Pa & subglacial (pore) water pressure \\
    \texttt{cbar} & m/year & magnitude of vertically-integrated horizontal velocity of ice \\
    \texttt{cbase} & m/year & magnitude of horizontal velocity of ice at base of ice \\
    \texttt{cflx} & m$^{2}$/year & magnitude of vertically-integrated horizontal flux of ice \\
    \texttt{csurf} & m/year & magnitude of horizontal velocity of ice at ice surface \\
    \texttt{dHdt} & m/year & rate of change of ice thickness \\
    \texttt{dbdt} & m/year& bedrock uplift rate \\
    \texttt{dhdt} & m/year& rate of change of surface elevation \\
    \texttt{enthalpybase} & K & ice enthalpy at the ice base\\
    \texttt{enthalpysurf} & K & ice enthalpy at the ice surface\\
    \texttt{hardav} & Pa s$^{1/n}$ & vertical average of ice hardness \\
    \texttt{lat} & degrees north& latitude \\
    \texttt{litho\und temp} & K & lithosphere (bedrock) temperature\\
    \texttt{lon} & degrees east & longitude \\
    \texttt{mask} & none & grounded/dragging/floating integer mask \\
    \texttt{strainheat} & mW/m$^{3}$ & rate of strain heating in ice (dissipation heating) \\
    \texttt{SigmaComp} & mW/m$^{3}$ & rate of compensatory strain heating in ice (PISMV only)\\
    \texttt{tauc} & Pa & yield stress for basal till (plastic or pseudo-plastic model) \\
    \texttt{taud} & Pa & magnitude of driving shear stress at base of ice \\
    \texttt{temp} & K & ice temperature \\
    \texttt{temp\und pa} & $^{\circ}$C & pressure-adjusted ice temperature \\
    \texttt{tempbase} & K & ice temperature at the ice base\\
    \texttt{tempsurf} & K & ice temperature at the ice surface\\
    \texttt{thk} & m & ice thickness\\
    \texttt{tillphi} & degrees & friction angle for till under grounded ice sheet \\
    \texttt{topg} & m & bedrock surface elevation \\
    \texttt{ubar} & m/year & vertical mean of horizontal ice velocity in the X direction \\
    \texttt{ub} & m/year & basal ice velocity in the X direction \\
    \texttt{usurf} & m & ice upper surface elevation \\
    \texttt{uvel} & m/year & horizontal velocity of ice in the X direction \\
    \texttt{uvelbase} & m/year & horizontal velocity of ice in the X direction
    at the ice base\\
    \texttt{uvelsurf} & m/year & horizontal velocity of ice in the X direction
    at the ice surface\\
    \texttt{vbar} & m/year & vertical mean of horizontal ice velocity in the Y direction \\
    \texttt{vvelbase} & m/year & horizontal velocity of ice in the Y direction
    at the ice base\\
    \texttt{vvelsurf} & m/year & horizontal velocity of ice in the Y direction
    at the ice surface\\
    \texttt{vb} & m/year & basal ice velocity in the Y direction \\
    \texttt{vvel} & m/year & horizontal velocity of ice in the Y direction \\
    \texttt{wvel} & m/year & vertical velocity of ice \\
    \texttt{wvelbase} & m/year &vertical velocity of ice at the ice base\\
    \texttt{wvelsurf} & m/year & horizontal velocity of ice at the ice surface\\
    \hline
  \end{longtable}
\normalsize
 \label{tab:extra-vars}
\end{table}

\clearpage

\subsection{Saving re-startable snapshots of the model state}\index{snapshots of the model state}\index{PISM!saving snapshots of the model state}
\label{sec:snapshots}  Sometimes you want to check the model state every 1000 years, for example.  One possible solution is to run PISM for a thousand years, have it save all the fields at the end of the run, then restart and run for another thousand, and etc.  This forces the adaptive time-stepping mechanism to stop \emph{exactly} at multiples of 1000 years, which may be desirable in some cases.

If saving exactly at specified times is not critical, then use the \intextoption{save\und file} and \intextoption{save\und times} options.  For example,
\begin{verbatim}
$ pismr -i foo.nc -y 10000 -o output.nc -save_file snapshots.nc \
        -save_times 1000:1000:10000
\end{verbatim}
starts a PISM evolution run, initializing from \verb|foo.nc|, running for
10000 years and saving snapshots to \verb|snapshots.nc| at the first time-step
after each of the years 1000, 2000, \dots, 10000.

We use a MATLAB-style range specification, $a:\Delta t:b$, where $a,\Delta t,b$ are in years.  The time-stepping scheme is not affected, but as a consequence we do not guarantee producing the exact number of snapshots requested if the requested save times have spacing comparable to the model time-steps.  This is not a problem in the typical case in which snapshot spacing is much greater than the length of a typical timestep.

It is also possible to save snapshots at intervals that are not equally-spaced
by giving the \verb|-save_times| option a comma-separated list. For example,
\begin{verbatim}
$ pismr -i foo.nc -y 10000 -o output.nc -save_file snapshots.nc \
        -save_times 1000,1500,2000,5000
\end{verbatim}
will save snapshots on the first time-step after years 1000, 1500, 2000 and 5000.
The comma-separated list given to the \verb|-save_times| option can be at most 200 numbers long.

If \verb|snapshots.nc| was created by the command above, running
\begin{verbatim}
$ pismr -i snapshots.nc -y 1000 -o output_2.nc
\end{verbatim}
will initialize using the last record in the file, at about $5000$ years.  By contrast, to restart from $1500$ years (for example) it is necessary to extract the corresponding record using \verb|ncks|\index{NCO (NetCDF Operators)!ncks}
\begin{verbatim}
$ ncks -d t,1500years snapshots.nc foo.nc
\end{verbatim}
and then restart from \verb|foo.nc|.  Note that \verb|-d t,N| means ``extract the $N$-th record'' (counting from zero).  So, this command is equivalent to
\begin{verbatim}
$ ncks -d t,1 snapshots.nc foo.nc
\end{verbatim}
Also note that the second snapshot will probably be \emph{around} $1500$ years and \verb|ncks| handles this correctly: it takes the record closest to $1500$ years.

Another possible use of snapshots is for restarting runs on a batch system which kills jobs which go over their allotted time.  Running PISM with options \verb|-y 1500| \verb|-save_times 1000:100:1400| would mean that if the job is killed before completing the whole 1500 year run, we can restart from near the last multiple of $100$ years.  Restarting with option \intextoption{ye} would finish the run on the desired year.

It is also possible to save snapshots to separate files using the
\intextoption{split\und snapshots} option.  For example, the run above can be changed to
\begin{verbatim}
$ pismr -i foo.nc -y 10000 -o output.nc -save_file snapshots \
        -save_times 1000,1500,2000,5000 -split_snapshots
\end{verbatim}
for this purpose.  This will produce files called \verb|snapshots-|year\verb|.nc|.  This option is generally faster if many snapshots are needed, apparently because of the time necessary to reopen a large file at each snapshot when \verb|-split_snapshots| is not used.  Note that tools like NCO\index{NCO (NetCDF Operators)!wildcards} and \verb|ncview|\index{ncview!wildcards} usually behave as desired with wildcards like ``\verb|snapshots-*.nc|''.

Table \ref{tab:snapshot-opts} lists the options related to saving snapshots of the model state.

\begin{table}[ht]
  \centering
  \caption{Command-line options controlling saving snapshots of the model state.}
  \begin{tabular}{p{0.35\linewidth}p{0.55\linewidth}}\hline
    \textbf{Option} & \textbf{Description} \\
    \hline
    \intextoption{save\und file} [file name] & Specifies the file to save to.\\
    \intextoption{save\und times} [range or list] & Specifies times at which to save snapshots, by either a MATLAB-style range $a:\Delta t:b$ or a comma-separated list. \\
    \intextoption{split\und snapshots} & Separate the snapshot output into files named \texttt{snapshots-}year\texttt{.nc}.  Faster if you are saving more than a dozen or so snapshots. \\
    \hline
  \end{tabular}
 \label{tab:snapshot-opts}
\end{table}


\subsection{PISM's default flags and parameters (and how to change them)}
\label{sec:pism-defaults}

PISM's behavior depends on values of many flags and physical parameters. Most of them have default values\footnote{For \texttt{pismr}, grid parameters $Mx$, $My$, $Mz$, $Mbz$, $Lz$, $Lbz$, that have to be set at bootstrapping, are some exceptions.} which are read from the configuration file \verb|pism_config.nc| in the \verb|lib| sub-directory.

It is possible to run PISM with an alternate configuration file using the \intextoption{config} command-line option:
\begin{verbatim}
$ pismr -i foo.nc -y 1000 -config my_config.nc
\end{verbatim}

The file \verb|my_config.nc| has to contain \emph{all the flags and parameters} present in \verb|pism_config.nc|.

The list of parameters is too long to include here; please see the PISM Class browser for an automatically-generated table describing them.

% FIXME: needs more work


\subsection{Run-time diagnostic viewers}
\label{sec:diagnostic-viewers}
Basic graphical views of the changing state of a PISM ice model are available at the command line by using options listed in table \ref{tab:diag-viewers}.
All the quantities listed in table \ref{tab:extra-vars} are available.

Note that (for performance and implementation reasons) map and slice viewers
are transposed.

\begin{table}[ht]
  \caption{Options controlling run-time diagnostic viewers}
  \centering
  \begin{tabular}{p{0.4\linewidth}p{0.5\linewidth}}\hline
    \small
   \textbf{Option} & \textbf{Description}\\
    \hline
    \intextoption{view\und map} [list of variables] & Turns on map-plane views of one or several variables, see tables \ref{tab:extra-vars}  \\
    \intextoption{view\und slice} [list of variables] & Turns on viewers showing a slice of a 3D field at a certain level above the base of ice\\
    \intextoption{slice\und level} number& Specifies the level for the previous option\\
    \intextoption{viewer\und size} number & desired viewer size, in pixels\\
    \intextoption{view\und sounding} [list of variables] &Turns on sounding viewers showing values in a column\\
    \intextoption{id} number & Sounding row-index\\
    \intextoption{jd} number & Sounding column-index\\
    \hline
  \normalsize
  \end{tabular}
 \label{tab:diag-viewers}
\end{table}
The option \verb|-view_map| shows map-plane views of 2D fields and surface views of 3D fields; for example:
\begin{verbatim}
$ pismr -i input.nc -y 1000 -o output.nc -view_map thk,temp 
\end{verbatim}
shows ice thickness and ice temperature at the surface.

It is possible to view 3D fields \emph{at a certain level above the base of the ice}:
\begin{verbatim}
$ pismr -i input.nc -y 1000 -o output.nc -view_slice temp -slice_level 1000
\end{verbatim}
shows the view of the ice temperature at 1000 meters above the base. (This will show air temperature in areas where ice thickness is less than 1000 meters.)

The option \verb|-view_sounding| allows viewing a \emph{sounding} of a variable at a given grid point:
\begin{verbatim}
$ pismr -i input.nc -y 1000 -o output.nc -view_sounding temp
\end{verbatim}
shows the dependence of the ice temperature on $z$ (more precisely, on the
number of the vertical grid layer) at the center of the domain. One can use options \verb|-id| and \verb|-jd| to choose a different point.

Some diagnostic quantities are \emph{only} available as run-time viewers:
\begin{center}
  \begin{tabular}{p{0.35\linewidth}p{0.55\linewidth}}\hline
    \small
    \textbf{Variable name or an option} & \textbf{Description}\\\hline
    \texttt{diffusivity} & diffusivity coefficient D in mass balance equation in $m^{2}/s$. Meaningful only in regions of shallow ice flow.\\
    \texttt{nuH} & two viewers: vertically-averaged effective viscosity times thickness; on $i$ and $j$ offset grids. Only meaningful in ice streams and shelves.\\
    \texttt{log\und nuH} & log base ten of the vertically-averaged effective viscosity times thickness on the regular grid. Only meaningful in ice streams and shelves.\\
    \intextoption{ksp\und monitor\und draw} & Iteration monitor for the Krylov subspace routines (KSP) in PETSc. Shows norm of residual versus iteration number.\\
    \normalsize
  \end{tabular}
\end{center}

\subsection{Managing parameter studies}
\label{sec:parameter-studies}
Keeping all PISM output files in a parameter study straight can be a challenge.

If parameters of interest can be controlled using command-line options, one can use \verb|ncdump -h| and look at the \verb|history| global attribute.

Changing parameters not available at the command-line is also possible~--- by using a custom configuration file (see section \ref{sec:pism-defaults}). The problem is that this makes it hard to see which parameters were changed.

The \intextoption{config\und override} command-line option provides an alternative. First of all, a file used with this option can have \emph{a subset} of flags and parameters present in \verb|pism_config.nc|. Moreover, PISM adds the \verb|pism_overrides| variable present in this file to the output file, making it easy to see which parameters were used to produce it.

Here's an example: suppose we want to compare the dynamics of an ice-sheet on Earth to the same ice-sheet on Mars.\footnote{This is an admittedly simplified example. One will surely need to change more than just the acceleration due to gravity for a real comparison.}

Running
\begin{verbatim}
$ pismr -i input.nc -y 1e5 -o earth.nc <other PISM options>
\end{verbatim}
produces the ``Earth'' result, since PISM's defaults correspond to this planet.

Next, we create \verb|mars.cdl|, containing the following:
\small
\begin{verbatim}
netcdf mars {
    variables:
    byte pism_overrides;
    pism_overrides:standard_gravity = 3.728;
    pism_overrides:standard_gravity_doc = "m s-2; standard gravity on Mars";
}
\end{verbatim}
\normalsize
Notice that the variable name is \verb|pism_overrides| and not \verb|pism_config| above. Now
\begin{verbatim}
$ ncgen -o mars_config.nc mars.cdl
$ pismr -i input.nc -y 1e5 -config_override mars_config.nc \
        -o mars.nc <other PISM options>
\end{verbatim}
will create \verb|mars.nc|, the result of the ``Mars'' run.

In this case, we can use \verb|ncdump| to see what was different about \verb|mars.nc|:
\small
\begin{verbatim}
$ ncdump -h mars.nc |grep pism_overrides
 byte pism_overrides ;
    pism_overrides:standard_gravity_doc = "m s-2; standard gravity on Mars" ;
    pism_overrides:standard_gravity = 3.728 ;
macbook:pism>
\end{verbatim}
\normalsize


\subsection{Visualizing climate inputs, without ice dynamics}\label{subsect:pcctest}  Recall that
internally in PISM there is a separation of climate inputs from ice dynamics
(subsection \ref{sec:climate-inputs}).  Executable \verb|pcctest|\index{PISM!pcctest}
allows one to visualize climate inputs without invoking the ice dynamics core of PISM.
This is helpful during the process of creating PISM-readable data files, and modeling 
with such.

FIXME: this section needs and overhaul!

\verb|pcctest| takes options listed in table \ref{tab:pcctest}.

\begin{table}[ht]
  \caption{\texttt{pcctest} command-line options}
  \centering
  \begin{tabular}{p{0.1\linewidth}p{0.8\linewidth}}\hline
    \textbf{Option} & \textbf{Description}\\
    \hline
    \texttt{-i} & specifies an input file, which has been produced by PISM;\\
    \texttt{-o} & sets the output file name;\\
    \texttt{-ca} &chooses the constant atmosphere coupler (\verb|PISMConstAtmosCoupler|);\\
    \texttt{-sma} & chooses the PDD atmosphere coupler (\verb|PISMPDDCoupler|);\\
    \texttt{-co} & chooses the constant ocean coupler (\verb|PISMConstOceanCoupler|);\\
    \texttt{-ys} & sets the start time, in years;\\
    \texttt{-ye} & sets the final time, in years;\\
    \texttt{-dt} & sets the interval between saved climate snapshots.\\
    \hline
 \end{tabular}
 \label{tab:pcctest}\index{\texttt{pcctest} command-line options}
\end{table}

\bigskip
As an example, set up an ice sheet state file and run \verb|pcctest| on it:
\begin{verbatim}
$ mpiexec -n 2 pisms -eisII A -y 10000 -o state.nc
$ pcctest -i state.nc -ca -ys 0.0 -ye 2.5 -dt 0.1 -o camovie.nc
\end{verbatim}
Using \verb|pisms| merely generates demonstration climate data, using
EISMINT II choices \cite{EISMINT00}.  \verb|pcctest| extracts the 
surface mass balance \verb|acab| and surface temperature \verb|artm| from \verb|state.nc|.
It then does nothing interesting, exactly because a constant climate
is used.  Viewing \verb|camovie.nc| we see these same values as from \verb|state.nc|,
in variables \verb|acab|, \verb|artm|, reported back to us as the time- and space-dependent
climate at times \verb|ys:dt:ye|.  It is a boring ``movie.''

The excuse for the executable \verb|pcctest| becomes clearer if we use a positive degree-day
model (subsection \ref{sec:boundary-models}).  The positive degree-day
model uses a variable called \verb|snowaccum|, and a calculation of melting, to get the
surface mass balance \verb|acab|.  As a demonstration here we use an NCO\index{NCO (NetCDF Operators)!ncrename} (subsection 
\ref{subsect:nctoolsintro}) to change a variable name in \verb|state.nc|.  Then we run \verb|pcctest|
with option \verb|-pdd| to invoke the \verb|PISMPDDCoupler|
object, which includes the parameterized PDD:
\begin{verbatim}
$ ncrename -v acab,snowaccum state.nc restate.nc
$ pcctest -i restate.nc -pdd -ys 0.0 -ye 2.5 -dt 0.1 -o pddmovie.nc
\end{verbatim}
In particular, a time-dependent view of the variable \verb|artmpdd| shows the time- and space-
dependent annual cycle.  Onto this is added a random perturbation (normally distributed with standard 
deviation of 5 degrees), but this random perturbation is not shown.  In fact the PDD uses the
\cite{CalovGreve05} method to compute the expected number of positive degree days.  From this
computation the net surface mass balance is computed, namely \verb|acab| in \verb|pddmovie.nc|.

See section \ref{sec:eismint-greenland} for another \verb|pcctest| example.

\subsection{Regridding}  It is common to want to interpolate a coarse grid model state onto a finer grid or vice versa.  For example, one might want to do the EISMINT II experiment on the default grid, producing output \verb|foo.nc|, but then interpolate both the ice thickness and the temperature onto a finer grid.  The basic idea of ``regridding'' in PISM is that one starts over from the beginning on the finer grid, but one extracts the desired variables stored in the coarse grid file and interpolates these onto the finer grid before proceeding with the actual computation.

The transfer from grid to grid is reasonably general---one can go from coarse to fine or vice versa in each dimension $x,y,z$---but the transfer must always be done by \emph{interpolation} and never \emph{extrapolation}.  (An attempt to do the latter will always produce a PISM error.)

Such ``regridding'' is done using the \intextoption{regrid\und file} and \intextoption{regrid\und vars} commands as in this example:\index{pisms}

\begin{verbatim}
$  pisms -eisII A -Mx 101 -My 101 -Mz 201 -y 1000 \
         -regrid_file foo.nc -regrid_vars thk,temp -o bar.nc
\end{verbatim}
\noindent By specifying regridded variables ``\verb|thk,temp|'', the ice thickness and temperature values from the old grid are interpolated onto the new grid.  Here one doesn't need to regrid the bed elevation, which is set identically zero as part of the EISMINT II experiment A description, nor the ice surface elevation, which is computed as the bed elevation plus the ice thickness at each time step anyway.

A slightly different use of regridding occurs when ``bootstrapping'', as described in section \ref{sect:boot} and illustrated by example in section \ref{sec:eismint-greenland}.

See table \ref{tab:regridvar} for the regriddable variables using
\intextoption{regrid\und file}.  Only model state variables are regriddable, while climate and boundary data generally are not explicitly regriddable.  (Bootstrapping, however, allows the same general interpolation as this explicit regrid.)

\begin{table}[ht]
\centering
\caption{Regriddable variables.\index{regrid}\index{regrid\und vars}  Use \texttt{-regrid\und vars} with these names.}\label{tab:regridvar}
\begin{tabular}{ll}\hline
\textbf{Name} & \textbf{Description}\\ \hline
\texttt{age} & age of ice\\
\texttt{bwat} & effective thickness of subglacial melt water \\
\texttt{dbdt} & bedrock uplift rate \\
\texttt{litho\und temp} & lithosphere (bedrock) temperature \\
\texttt{mask} & grounded/dragging/floating integer mask, see section \ref{subsect:floatmask} \\
\texttt{temp} & ice temperature \\
\texttt{thk} & land ice thickness \\
\texttt{topg} & bedrock surface elevation \\
\hline
\normalsize
\end{tabular}
\end{table}

Here is another example: suppose you have an output of a PISM run on a fairly coarse grid (stored in \verb|foo.nc|) and you want to continue this run on a finer grid. This can be done using \intextoption{regrid\und file} along with \intextoption{boot\und from}:
\begin{verbatim}
$ pismr -boot_from foo.nc -Mx 201 -My 201 -Mz 21 -Mbz 21 -Lz 4000 \
        -regrid_from foo.nc -regrid_vars litho_temp,age,temp -y 100 -o bar.nc
\end{verbatim}
In this case all the model-state 2D variables present in \verb|foo.nc| will be interpolated onto the new grid during bootstrapping, which happens first, while three-dimensional variables are filled using heuristics mentioned in section \ref{sect:boot}.  Then temperature in bedrock (\verb|litho_temp|)\footnote{Assumes that foo.nc has \t{Mbz}$>1$.}, age of the ice (\verb|age|) and ice temperature (\verb|temp|) will be interpolated from \verb|foo.nc| onto the new grid during the regridding stage, overriding values set at the bootstrapping stage.  All of this, bootstrapping and regridding, occurs before the first timestep.


\newcommand\pid{\textsl{pid}s}

\subsection{Signals, to control a running PISM model} \label{subsect:signal} \index{signals} \index{PISM!catches signals -TERM and -USR1}  Ice sheet model runs sometimes take a long time, so the state of a run may need checking.  Sometimes the run needs to be stopped, but with the possibility of restarting.  PISM implements these behaviors using ``signals'' from the POSIX standard, included in Linux and most flavors of Unix.  Table \ref{tab:signals} summarizes how PISM responds to signals.  A convenient form of \verb|kill|, for Linux users, is \verb|pkill| which will find processes by executable name.  Thus ``\verb|pkill -USR1 pismr|'' might be used to send all PISM processes the same signal, avoiding an explicit list of \pid.

\begin{table}[ht]
\caption{Signalling running PISM processes.  ``\pid''~stands for list of all identifiers of the PISM processes.}\label{tab:signals}
\begin{tabular}{p{0.20\linewidth}p{0.10\linewidth}p{0.70\linewidth}}\hline
\textbf{Command} & \textbf{Signal} & \textbf{PISM behavior} \\
\hline
\texttt{kill -KILL} \pid & \texttt{SIGKILL} & Terminate with extreme prejudice. PISM cannot catch it and no state is saved. \\
\texttt{kill -TERM} \pid & \texttt{SIGTERM} & End processes, but save the last model state in the output file, using \verb|-o| name or default name as normal.  Note that the \verb|history| string in the output file will contain an ``\verb|EARLY EXIT caused by signal SIGTERM|'' indication. \\
\texttt{kill -USR1} \pid & \texttt{SIGUSR1} & Allow process(es) to continue, but save the model state at the current time as ``\texttt{pism}\textsl{X}\texttt{-}\textsl{year}\texttt{.nc}''.  Time-stepping is not altered.  Also flushes time-series output buffers. \\
\texttt{kill -USR2} \pid & \texttt{SIGUSR2} & Just flush time-series output buffers. \\
\hline
\end{tabular}
\end{table}

Here is an example.  Suppose we start a long verification run in the background, with standard out redirected into a file:\index{pismv}

\verb|pismv -test G -Mz 101 -y 1e6 -o testGmillion.nc >> log.txt &|

\noindent This run gets a Unix process id,\index{signals!the pid}, which we assume is ``8920''.  (Get it using \verb|ps| or \verb|pgrep|.)  If we want to observe the run without stopping it we send the \verb|USR1| signal:\index{signals!usr1}

\verb|kill -USR1 8920|

\noindent Suppose it happens that we caught the run at year 31871.5.  Then, for example, a NetCDF file \verb|pismv-31871.495.nc| is produced.  Note also that in the standard out log file \verb|log.txt| the line

\begin{verbatim}
caught signal SIGUSR1:  Writing intermediate file ... and flushing time series.
\end{verbatim}
\noindent appears around that time step.  Suppose, on the other hand, that the run needs to be stopped.  Then a graceful way is\index{signals!term}

\verb|kill -TERM 8920|

\noindent because the model state is saved and can be inspected.



\subsection{Understanding adaptive time-stepping} \label{subsect:adapt}\index{PISM!adaptive time stepping scheme} At each time step the PISM standard output includes ``flags'' and then a summary of the model state using a few numbers.  A typical example is
\small
\begin{verbatim}
 y SIA v$th d (dt=1.07878 in 2 substeps; av dt_sub=0.53939)
  SSA:     4 outer iterations
S  -3456.41429:  3.35908  2.1664   2748.786  271.2114
\end{verbatim}
\normalsize
\noindent The format of the third line, the summary, is simple:
\small
\begin{verbatim}
S         YEAR:     ivol   iarea     thick0     temp0
\end{verbatim}
\normalsize
That is, we have the total ice volume, total ice area, and the thickness and basal temperature at the center of the computational domain.  A ``\verb|U|'' line near the beginning of the standard output indicates the units.

The characters ``\verb| y SIA v$th|'' at the beginning of the flag line give a very terse description of which physical processes were modeled in that time step.  Both the SIA and SSA stress balances were solved, and the latter, which is the most computationally-expensive, describes its nonlinear iterations.

Now we explain what the rest of the flags line, namely

``\verb|d (dt=1.07878 in 2 substeps; av dt_sub=0.53939)|''

\noindent might mean.  Note that the PISM time step is determined by an adaptive mechanism.  PISM does each step explicitly when numerically-approximating mass conservation in the map-plane.  This, and the modeling of horizontal advection in conservation of energy and age evolution, requires such an adaptive time-stepping scheme \cite{BBL}.  The first character we see here, namely ``\verb|d|'', is the adaptive-timestepping ``reason'' flag.  See Table \ref{tab:adaptiveflag}; ``\verb|d|'' means that the time step was limited by the diffusivity of SIA mass conservation.  We also see that there was a major time step of 1.08 model years divided into 2 substeps of about 0.54 years.  The \intextoption{skip} option sets the maximum number of such substeps, but the adaptive mechanism may choose to take fewer substeps to satisfy numerical stability criteria.

\begin{table}[ht]
\caption{Meaning of the adaptive time-stepping ``reason'' flag in the standard output flag line.}\label{tab:adaptiveflag}
\begin{tabular}{p{0.05\linewidth}p{0.85\linewidth}}\hline
\textbf{Flag} & \textbf{Active adaptive constraint} \\ \hline
\verb|c| & three-dimensional CFL for temperature/age advection \cite{BBL} \\
\verb|d| & diffusivity for SIA mass conservation \cite{BBL,HindmarshPayne} \\
\verb|e| & end of prescribed run time \\
\verb|f| & \verb|-dt_force| set; generally option \verb|-dt_force|, which overrides the adaptive scheme; \emph{should not be used}  \\
\verb|m| & maximum allowed $\Delta t$ applies; set with \verb|-max_dt| \\
\verb|t| & maximum $\Delta t$ was temporarily set by a derived class, or by the mechanism which saves time-series of spatially-varying quantities \\
\verb|u| & 2D CFL for mass conservation in SSA regions (upwinded; \cite{BBssasliding})\\
\hline
\normalsize
\end{tabular}
\end{table}


\subsection{Utility and test scripts} \label{subsect:scripts}\index{python scripts} Some nice scripts.
 
FIXME: the python scripts are all documented at the \textsl{Packages} tab on the browser

FIXME: the python scripts all take \texttt{fill\und missing.py --help}

\newcommand{\scripthead}[1]{\texttt{#1}}

\begin{table}[h]
  \centering
  \begin{tabular}{p{0.35\linewidth}p{0.65\linewidth}}\\
    \hline
    \textbf{Script} & \textbf{Function(s)}\\
    \hline
    \scripthead{test/vfnow.py} & Organizes the process of verifying PISM.  Specifies standard refinement paths for each of the tests (section \ref{sect:verif}). \\
    \scripthead{test/vnreport.py} & Automates the creation of convergence graphs like figures \ref{fig:thickerrsB}--~\ref{fig:temperrsK}. \\
    \scripthead{util/fill\und missing.py} & Uses an approximation to Laplace's equation $\grad^2 u = 0$ to smoothly replace missing values in a two-dimensional NetCDF variable.  The ``hole'' is filled with an average of the boundary non-missing values. Section \ref{sec:eismint-greenland} gives an example application to the bed elevations in the EISMINT-Greenland data.  Depends on \verb|netcdf4-python| and \verb|scipy| Python packages. \\
    \scripthead{util/check\und polarstereo.py} & \\
    \scripthead{util/check\und stationarity.py} & \\
    \scripthead{util/nc2mat.py} & Reads specified variables from a NetCDF file and writes them to an output file in the MATLAB binary data file format \verb|.mat|, supported by MATLAB version 5 and later.  Depends on \verb|netcdf4-python| and \verb|scipy| Python packages. \\
    \scripthead{util/nccmp.py} & \\
    \scripthead{util/pism\und config\und editor.py} & \\
    \scripthead{util/pism\und matlab.m} & \\
   \hline
  \end{tabular}
  \caption{Some scripts for helping PISM users.}
  \label{tab:scripts-overview}
\end{table}



%%% Local Variables: 
%%% mode: latex
%%% TeX-master: "manual"
%%% End: 
