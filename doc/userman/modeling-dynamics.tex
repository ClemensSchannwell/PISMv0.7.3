
\section{Modeling choices:  Ice dynamics and thermodynamics}
\label{sec:modeling-dynamics}

\subsection{Ice rheology}
\label{sec:rheology}
\optsection{Ice rheology}

The ``rheology'' of a viscous fluid refers to the relation between the applied stress and the resulting deformation, the strain rate.  The models of ice rheology available in PISM are all of isotropic Glen-Nye type \cite{Paterson}.   A rheology in this class is described by a ``flow law'', which is a function $F(\sigma,T,\omega,P,d)$ in the ``constitutive relation''
\begin{equation}
D_{ij} = F(\sigma,T,\omega,P,d)\, \sigma_{ij}'.  \label{eq:constitutive}
\end{equation}
Here $D_{ij}$ is the strain rate tensor, $\sigma_{ij}'$ is the stress deviator tensor, $T$ is the ice temperature, $\omega$ is the liquid water fraction, $P$ is the pressure, $d$ is the grain size, and $\sigma^2 = \frac{1}{2} \|\sigma_{ij}'\|_F = \frac{1}{2} \sigma_{ij}' \sigma_{ij}'$ so $\sigma$ is the second invariant of the stress deviator tensor.

For example,
\begin{equation}
F(\sigma,T) = A(T) \sigma^{n-1}  \label{eq:isothermalglen}
\end{equation}
is the common temperature-dependent Glen law \cite{PatersonBudd,BBL}, which has no dependence on liquid water fraction, pressure, or grain size.  If the ice softness $A(T)=A_0$ is constant then the law is isothermal, whereas if there is dependence on temperature then $A(T)$ is usually a generalization of ``Arrhenius'' form $A(T) = A \exp(-Q/RT^*)$.  The more elaborate Goldsby-Kohlstedt law \cite{GoldsbyKohlstedt} is also a function $F(\sigma,T,P,d)$.

In the enthalpy mode of PISM, which is the thermodynamical modeling default, there is only one choice for the flow law: the Glen-Paterson-Budd-Lliboutry-Duval law \cite{AschwandenBuelerKhroulevBlatter,LliboutryDuval1985,PatersonBudd}, a function $F(\sigma,T,\omega,P)$.  This law is the only one in the literature that depends on both the temperature and the liquid water fraction; it parameterizes the (observed) softening of pressure-melting-temperature ice as its liquid fraction increases.  One can use this default polythermal law or one may choose among a number of ``cold ice'' laws listed in Table \ref{tab:flowlaw} which do not use the liquid water fraction.  

Command-line options \intextoption{sia_flow_law} and \intextoption{ssa_flow_law} control which flow law is used by the SIA and SSA ice dynamics models, respectively.  Allowed arguments are listed in table \ref{tab:flowlaw} below.  Form \eqref{eq:constitutive} of the flow law is used in the SIA.  However, the ``viscosity'' form of such a flow law, found by inverting the constitutive relation, is needed for ice shelf and ice stream (SSA) flow \cite{BBssasliding}:
	$$\sigma_{ij}' = 2 \nu(D,T,\omega,P,d)\,D_{ij} $$
Here $\nu(D,T,\omega,P,d)$ is the ``effective viscosity'' and $D^2 = \frac{1}{2} D_{ij} D_{ij}$.  Viscosity form is not known for the Goldsby-Kohlstedt law \cite{GoldsbyKohlstedt}, so option ``\texttt{-ssa_flow_law gk}'' is an error.

One can also choose either \texttt{-sia_flow_law isothermal_glen} or \texttt{-ssa_flow_law isothermal_glen}, which is $F(\sigma) = A_0 \sigma^{n-1}$ with inverse $\nu(D) = \frac{1}{2} B_0 D^{(1-n)/(2n)}$ where $A_0$ is the ice softness and $B_0=A_0^{-1/n}$ is the ice hardness.

Options \intextoption{sia_e} and \intextoption{ssa_e} set flow enhancement factors for the SIA and SSA respectively. These options can be used with any flow law.  Option \texttt{-sia_e} sets ``$e$'' in ``$D_{ij} = e\, F(\sigma,T,\omega,P,d)\, \sigma_{ij}',$'' in equation \eqref{eq:constitutive}.  Option \texttt{-ssa_e} sets ``$e$'' in the viscosity form so that ``$\sigma_{ij}'  = e^{-1/n}\, 2\, \nu(D,T,\omega,P,d)\, D_{ij}.$''

Flow law parameters such as ice softness can be changed using configuration parameters (see section \ref{sec:pism-defaults} and the implementation of flow laws in the \emph{Source Code Browser}).  One can also choose the scalar function $F$ reasonably arbitrarily by modifying source code; see source files \texttt{flowlaws.hh}, \texttt{flowlaws.cc} in \texttt{src/base/rheology/}; note that Table \ref{tab:flowlaw} also lists the C++ classes declared in \texttt{flowlaw.hh}.

\begin{table}[ht]
\centering
\index{rheology}\index{flow law}
\small
\begin{tabular}{p{0.16\linewidth}p{0.2\linewidth}p{0.58\linewidth}}\toprule
\textbf{Type} & C++ Class & \textbf{Comments and Reference} \\ \midrule
\texttt{gpbld} &\texttt{GPBLDIce}  & Glen-Paterson-Budd-Lliboutry-Duval law \cite{LliboutryDuval1985}, the enthalpy-based default in PISM \cite{AschwandenBuelerKhroulevBlatter}.  Extends the Paterson-Budd law (below) to positive liquid water fraction.  If $A_{c}(T)$ is from Paterson-Budd then this law returns $A(T,\omega) = A_{c}(T) (1 + C \omega)$, where $\omega$ is the liquid water fraction, $C$ is a configuration parameter \texttt{gpbld_water_frac_coeff} [default $C=181.25$], and $\omega$ is capped at level \texttt{gpbld_water_frac_observed_limit}.  \\  \midrule
\texttt{pb} &\texttt{ThermoGlenIce}  & Paterson-Budd law, the cold-mode default.  Fixed Glen exponent $n=3$.  Has a split ``Arrhenius'' term $A(T) = A \exp(-Q/RT^*)$ where \mbox{$A = 3.615 \times 10^{-13}\, \text{s}^{-1}\, \text{Pa}^{-3}$}, \mbox{$Q = 6.0 \times 10^4\, \text{J}\, \text{mol}^{-1}$} if $T^* < 263$ K and
 \mbox{$A = 1.733 \times 10^{3}\, \text{s}^{-1}\, \text{Pa}^{-3}$}, \mbox{$Q = 13.9 \times 10^4\, \text{J}\, \text{mol}^{-1}$} if $T^* > 263$ K; here $T^*$ is pressure-adjusted temperature \cite{PatersonBudd}. \\
\texttt{arr} &  \texttt{ThermoGlenArrIce} & \emph{Cold} part of Paterson-Budd.  Regardless of temperature, the $A$ and $Q$ values for $T^*<263$ K in  the Paterson-Budd law apply.  This is the flow law used in the thermomechanically coupled exact solutions Tests \textbf{F} and \textbf{G} described in \cite{BBL,BB} and run by \texttt{pismv -test F} and \texttt{pismv -test G}. \\
\texttt{arrwarm} & \texttt{ThermoGlenArrIceWarm} & \emph{Warm} part of Paterson-Budd.  Regardless of temperature, the $A$ and $Q$ values for $T^*>263$ K in Paterson-Budd apply.\\
\texttt{hooke} & \texttt{HookeIce} & Hooke law \mbox{$A(T) = A \exp(-Q/(RT^*) + 3C (T_r - T^*)^\kappa)$.}  Fixed Glen exponent $n=3$ and constants as in \cite{Hooke,PayneBaldwin}.\\
\texttt{gk} & \texttt{GoldsbyKohlstedtIce} & The  Goldsby-Kohlstedt flow law.  This law has a combination of exponents  from $n=1.8$ to $n=4$ \cite{GoldsbyKohlstedt}. It does not have a viscosity form and can only be used by the SIA stress balance. \\
\texttt{isothermal_glen} &  \texttt{IsothermalGlenIce} &The isothermal Glen flow law. \\
\bottomrule
\normalsize	
\end{tabular}
\caption{Choosing the ice rheology using \texttt{-sia_flow_law} and \texttt{-ssa_flow_law}.  Flow law choices other than \texttt{gpbld} do not use the liquid water fraction $\omega$ but only the temperature $T$.}
\label{tab:flowlaw}
\end{table}


\clearpage

\subsection{Choosing the stress balance}
\label{subsect:stressbalance}
\optsection{Stress balance}

The basic stress balance used for all grounded ice in PISM is the non-sliding, thermomechanically-coupled SIA \cite{BBL}.  For the vast majority of most ice sheets, as measured by area or volume, this is an appropriate model, which is an $O(\eps^2)$ approximation to the Stokes model if $\eps$ is the depth-to-length ratio of the ice sheet \cite{Fowler}.

The shallow shelf approximation (SSA)\index{SSA (shallow shelf approximation)} stress balance applies to floating ice.  See the Ross ice shelf example in section \ref{sec:ross} for an example in which the SSA is only applied to floating ice.

The SSA is also used in PISM to describe the sliding of grounded ice and the formation of ice streams \cite{BBssasliding}.  Specifically for the SSA with ``plastic'' (Coulomb friction) basal resistance, the locations of ice streams are determined as part of a free boundary problem of Schoof \cite{SchoofStream}, a model for emergent ice streams within a ice sheet and ice shelf system.  This model explains ice streams through a combination of plastic till failure and SSA stress balance.  As a combination of SIA and SSA it is a ``hybrid'' approximation of the Stokes model \cite{BBssasliding,Winkelmannetal2011}.  In other words, this SSA description of ice streams is the preferred ``sliding law'' for the SIA, and it should be used in preference to classical SIA sliding laws which make ice basal velocity a local function of the basal value of the driving stress \cite{BBssasliding}.\index{SIA (shallow ice approximation)!sliding laws}

Option \texttt{-stress_balance ssa+sia} turns on such use of the SSA as a sliding law using the ``plastic'' basal resistance law; floating ice is also subject to the SSA with this option.  Of course there is more to the use of a stress balance than just turning it on!  At all grounded points a yield stress, or a pseudo-yield-stress in the case of power law sliding (subsection \ref{subsect:basestrength}), is computed from the amount of stored basal water and from a (generally) spatially-varying till strength.  The amount of stored basal water is modeled by the subglacial hydrology mode choice (subsection \ref{subsect:subhydro}) based on the basal melt rate which is, primarily, thermodynamically-determined (subsection \ref{subsect:basestrength}).

Table \ref{tab:stressbalchoice} describes the basic choice of stress balance, while Table \ref{tab:ssausage} describes additional controls on the numerical solution of the stress balance equations.  If the ice sheet being modeled has any floating ice then the user is advised to read section \ref{sec:pism-pik} on modeling marine ice sheets.

\begin{table}[ht]
\centering
\small
\begin{tabular}{p{0.25\linewidth}p{0.7\linewidth}}
\toprule
\textbf{Option} & \textbf{Semantics}\\ \midrule
  \intextoption{stress_balance none} & Turn off ice flow completely.\\
  \intextoption{stress_balance sia} \mbox{(default)} & Grounded ice flows by the non-sliding SIA.  Floating ice essentially doesn't flow, so this model is not recommended for marine ice sheets. \\
  \intextoption{stress_balance ssa} & Use the SSA model exclusively. Horizontal ice velocity is constant throughout ice columns.\\
  \intextoption{stress_balance \mbox{prescribed_sliding}} & Use the constant-in-time prescribed sliding velocity field read from a file set using \fileopt{prescribed_sliding_file}, variables \texttt{ubar} and \texttt{vbar}. Horizontal ice velocity is constant throughout ice columns. \\
  \intextoption{stress_balance ssa+sia} & The recommended sliding law, which gives the SIA+SSA hybrid stress balance.  Combines SSA-computed velocity, using pseudo-plastic till, with SIA-computed velocity according to the combination in \cite{BBssasliding}.  Floating ice uses SSA only. \\
  \intextoption{stress_balance \mbox{prescribed_sliding+sia}} & Use the constant-in-time prescribed sliding velocity in combination with the non-sliding SIA.\\
\bottomrule
\end{tabular}
\normalsize
\caption{The basic choice of stress balance.}
\label{tab:stressbalchoice} 
\end{table}


\begin{table}[ht]
  \centering
  \begin{tabular}{p{0.25\linewidth}p{0.7\linewidth}}
     \toprule
     \textbf{Option} & \textbf{Description}\\\midrule
     \intextoption{ssa_method} [\texttt{fd}$\big|$\texttt{fem}] & Both finite difference (\texttt{fd}; the default) and finite element (\texttt{fem}) versions of the SSA numerical solver are implemented in PISM.  They behave similarly for runs without PIK options (section \ref{sec:pism-pik}), but the \texttt{fd} solver is the only one which allows PIK options.  \texttt{fd} uses Picard iteration \cite{BBssasliding}, while \texttt{fem} uses a Newton method.  The \texttt{fem} solver has surface velocity inversion capability \cite{Habermannetal2013}.  \\
     \intextoption{ssa_eps} ($10^{13}$) & The numerical scheme for the SSA computes an effective viscosity $\nu$ which which depends on strain rates and ice hardness (thus on temperature).  The minimum value of the effective viscosity times the thickness (i.e.~$\nu H$) is important to the difficulty of solving the numerical SSA.  This constant is added ($\nu H \to \nu H + \text{\texttt{ssa_eps}}$) to keep $\nu H$ bounded away from zero.  The units of \texttt{ssa_eps} are $\text{Pa}\,\text{m}\,\text{s}$.  Turn off this lower bound by \texttt{-ssa_eps 0.0}.  Use option \texttt{-ssa_view_nuh} to view the product $\nu H$ for your simulation, to evaluate the relative importance of this \texttt{ssa_eps} regularization.  Note that in a typical Greenland run we see a wide range of values for $\nu H$ from $\sim 10^{14}$ to $\sim 10^{20}$ $\text{Pa}\,\text{m}\,\text{s}$. \\
     \intextoption{ssa_maxi} (300) & (\emph{Only active with} \texttt{-ssa_method fd}.)  Set the maximum allowed number of Picard (nonlinear) iterations in solving the shallow shelf approximation.\\
     \intextoption{ssa_rtol} ($10^{-4}$) & (\emph{Only active with} \texttt{-ssa_method fd}.)  The numerical scheme for the SSA does a nonlinear iteration wherein velocities (and temperatures) are used to compute a vertically-averaged effective viscosity which is used to solve the equations for horizontal velocity.  Then the new velocities are used to recompute an effective viscosity, and so on.  This option sets the relative change tolerance for the effective viscosity.  In particular, the nonlinear part of the iteration requires that successive values $\nu^{(k)}$ of the vertically-averaged effective viscosity satisfy
	$\|(\nu^{(k)} - \nu^{(k-1)}) H\|_1 \le \text{\texttt{ssa_rtol}} \|\nu^{(k)} H\|_1$
in order to end the iteration with $\nu = \nu^{(k)}$. \\
    \intextoption{ssafd_ksp_rtol} ($10^{-5}$) & (\emph{Only active with} \texttt{-ssa_method fd}.)  Set the relative change tolerance for the iteration inside the linear solver. \\
\bottomrule
\end{tabular}
\caption{Controlling the numerical SSA stress balance in PISM}
\label{tab:ssausage}
\end{table}


\clearpage

\subsection{Surface gradient method}
\label{subsect:gradient}
\optsection{Driving stress computation}

PISM computes surface gradients to determine the ``driving stress''
	$$(\tau_{d,x},\tau_{d,y}) = - \rho g H \grad h,$$
where $H$ is the ice thickness, and $h = H+b$ is the ice surface elevation.  The driving stress enters into both the SIA and SSA stress balances, but in the former the driving stress is needed on a staggered grid, while in the latter the driving stress is needed on the regular grid.

Surface gradients are computed by finite differences in several slightly-different ways.  There are options for choosing which method to use, but to the best of our knowledge there is no theoretical advice on the best, most robust mechanism.  There are three \intextoption{gradient} methods in PISM:

\noindent\texttt{-gradient mahaffy}\quad  This most ``standard'' way computes the surface slope onto the staggered grid for the SIA \cite{Mahaffy}.  It makes $O(\Delta x^2,\Delta y^2)$ errors.  For computations of driving stress on the regular grid, centered differencing is used instead.

\noindent\texttt{-gradient haseloff}\quad  This is the default method.  It only differs from \texttt{mahaffy} at ice-margin locations, where it alters the formula for the slope in cases where an adjacent ice-free bedrock surface elevation is above the ice elevation.

\noindent\texttt{-gradient eta}\quad  In this method we first transform the thickness $H$ by $\eta = H^{(2n+2)/n}$ and then differentiate the sum of the thickness and the bed using centered differences:
	$$\grad h = \grad H + \grad b = \frac{n}{(2n+2)} \eta^{(-n-2)/(2n+2)} \nabla \eta + \nabla b.$$
Here $b$ is the bed elevation and $h$ is the surface elevation.  This transformation sometimes has the benefits that the surface values of the horizontal velocity and vertical velocity, and the driving stress, are better behaved near the margin.  See \cite{BLKCB} for technical explanation of this transformation and compare \cite{SaitoMargin}.  The actual finite difference schemes applied to compute the surface slope are similar to option \texttt{mahaffy}.


\subsection{Modeling conservation of energy} \label{subsect:energy}
\optsection{Energy conservation}

In normal use PISM solves the conservation of energy problem within the ice, the thin subglacial layer, and a layer of thermal bedrock.  For the ice and the subglacial layer it uses an enthalpy-based scheme \cite{AschwandenBuelerKhroulevBlatter} which allows the energy to be conserved even when the temperature is at the pressure-melting point.

Ice at the melting point is called ``temperate'' ice.  Part of the thermal energy of temperate ice is in the latent heat of the liquid water stored between the crystals of the temperate ice.  Part of the thermal energy of the whole glacier is in the latent heat of the liquid water under the glacier.  The enthalpy scheme correctly models these storehouses of thermal energy, and thus it allows polythermal and fully-temperate glaciers to be modeled \cite{AschwandenBlatter}.

The state of the full conservation of energy model includes the 3D \texttt{enthalpy} variable plus the 2D \texttt{bwat} and \texttt{tillwat} subglacial hydrology state variables (subsection \ref{subsect:subhydro}), all of which are seen in output files.  The important basal melt rate computation involves all of these energy state variables, because the basal melt rate (\texttt{bmelt} in output files) comes from conserving energy across the ice-bedrock layer \cite{AschwandenBuelerKhroulevBlatter}.  Fields \texttt{temp}, \texttt{liqfrac}, and \texttt{temp_pa} seen in output files are all actually diagnostic outputs because all of these can be recovered from the enthalpy and the ice geometry.

Because this part of PISM is just a conservation law, there is little need for the user to worry about controlling it.  If desired, however, conservation of energy can be turned off entirely with \intextoption{energy none}.  The default enthalpy-based conservation of energy model (i.e.~\texttt{-energy enthalpy}) can be replaced by the temperature-based (i.e.~``cold ice'') method used in \cite{BBssasliding} and verified in \cite{BBL} by setting option \intextoption{energy cold}.

The thermal bedrock layer model is turned off by setting \texttt{-Mbz 1} (i.e.~zero spaces) while it is turned on by choosing a depth and number of points, as in \texttt{-Lbz 1000 -Mbz 21}, for example, which gives a layer depth of 1000 m and grid spaces of 50 m (= 1000/20).  The input geothermal flux (\texttt{bheatflx} in output files) is applied at the bottom of the bedrock thermal layer if such a layer is present and otherwise it is applied at the base of the ice.


\subsection{Computing ice age} \label{subsect:age}
\optsection{Computing ice age}

By default, PISM does not compute the age of the ice\index{PISM!modeling the age of the ice} because it does not directly impact ice flow when using the default flow laws.  It is very easy to turn on.  Just set \intextoption{age}.  A 3D variable \texttt{age} will appear in output files.  It is read at input if \texttt{-age} is set and otherwise it is ignored even if present in the input file.  If \texttt{-age} is set and the variable \texttt{age} is absent in the input file then the initial age is set to zero.


%%% Local Variables: 
%%% mode: latex
%%% TeX-master: "manual"
%%% End: 

% LocalWords:  
