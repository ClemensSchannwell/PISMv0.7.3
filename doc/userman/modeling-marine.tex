
\section{Modeling choices: Marine ice sheet modeling}
\label{sec:pism-pik}
\index{PISM!PISM-PIK}
\index{PISM!marine ice sheet modeling}
\optsection{Marine ice sheets}

PISM is often used to model whole ice sheets surrounded by ocean, with attached ice shelves floating on it, or smaller regions like outlet glaciers flowing into embayments.  This section explains the geometry and stress balance mechanisms in PISM that apply to floating ice and especially at the vertical calving face of floating ice shelves.

The physics at calving fronts is very different from elsewhere on an ice sheet, because the flow is nothing like the lubrication flow addressed by the SIA, and nor is the physics like the sliding flow in the interior of an ice domain, where sliding alters geometry only according to the basic mass continuity equation.  The physics at the calving front can be thought of as certain modifications to the mass continuity equation and to the SSA stress balance equation.

References \cite{Albrechtetal2011,Levermannetal2012,Winkelmannetal2011} describe most of the mechanisms covered in this section.  These are all improvements to the grounded, SSA-as-a-sliding law model of \cite{BBssasliding}.  These improvements make PISM an effective Antarctic model, and generally a continental marine ice sheet model, as demonstrated by \cite{Martinetal2011} and many PISM applications since then.  These improvements had a separate existence as the ``PISM-PIK'' model from 2009--2010.

A summary of options to turn on these PISM-PIK mechanisms, other than ``eigencalving'', is in Table \ref{tab:pism-pik-part-grid}.  When in doubt, the PISM user should simply set \intextoption{pik} to turn on all of them, but the user must still choose a calving model.  Subsections \ref{sec:part-grid}, \ref{sec:cfbc},  \ref{sec:kill-icebergs}, and \ref{sec:calving} below describe these PISM-PIK mechanisms in more detail.


\subsection{Flotation criterion and mask}
\label{sec:floatmask}

The most basic decision about marine ice sheet dynamics made internally by PISM is whether to apply the ``flotation criterion'' to determine whether the ice is floating on the ocean or not.  In an evolution run this decision is made at each time step and the result is stored in the \texttt{mask} variable.

The possible values of the \texttt{mask}\index{mask} are given in Table \ref{tab:maskvals}.  The mask does not by itself determine ice dynamics.  For instance, even when ice is floating (mask value \texttt{MASK_FLOATING}), the user must turn on the usual choice for ice shelf dynamics, namely the SSA stress balance, by either option \intextoption{ssa_floating_only} or \intextoption{ssa_sliding}.

\begin{table}[ht]
  \centering
 \small
  \begin{tabular}{p{0.25\linewidth}p{0.65\linewidth}}
    \toprule
    \textbf{Mask value} & \textbf{Meaning}\\
    \midrule
    -1=\texttt{MASK_UNKNOWN} & not set, does not appear in PISM output files \\
    0=\texttt{MASK_ICE_FREE_BEDROCK} & ice free bedrock \\
    2=\texttt{MASK_GROUNDED}& ice is grounded (and the SIA is applied unless \intextoption{no_sia}; the SSA is also applied if \texttt{-ssa_sliding} is given) \\
    3=\texttt{MASK_FLOATING} & ice is floating (the SIA is never applied; the SSA is applied if one of \texttt{-ssa_sliding} or \texttt{-ssa_floating_only} is given) \\
    4=\texttt{MASK_ICE_FREE_OCEAN} & ice-free ocean \\
    \\\bottomrule
  \end{tabular}
  \normalsize
  \caption{The PISM mask\index{mask}, in combination with user options, determines the dynamical model.}
  \label{tab:maskvals} 
\end{table}

Assuming the geometry of the ice evolves (which can be turned off by option \texttt{-no_mass}), and assuming an ocean exists so that a sea level is used in the flotation criterion (which can be turned off by option \intextoption{dry}), then at each time step the mask changes by the flotation criterion.  Ice which becomes floating is marked as \texttt{MASK_FLOATING} while ice which becomes grounded is marked as \texttt{MASK_GROUNDED}.


\subsection{Sea level}
\label{sec:sealevel}
FIXME: read it from a file with \texttt{-ocean FOO,delta_SL -ocean_delta_SL_file bar}

FIXME: config param or option?


\subsection{Partially-filled cells at the boundaries of ice shelves}
\label{sec:part-grid}
Albrecht et al \cite{Albrechtetal2011} argue that the correct movement of the ice shelf calving front on a finite-difference grid, assuming for the moment that ice velocities are correctly determined (see below), requires tracking some cells as being partially-filled (option \intextoption{part_grid}).  If the calving front is moving forward, for example, the at a time step the next cell gets a little ice.  It is not correct to add that little mass as a thin layer of ice which fills the cell.  (That would smooth the steep ice front after a couple of time steps.)  Instead the cell must be regarded as having ice which is comparably thick to the upstream cells, but where the ice only partly-fills the cell.

Specifically, the PISM-PIK mechanism turned on by \texttt{-part_grid} adds mass to the partially-filled cell which the advancing front enters, and it determines the coverage ratio according to the ice thickness of neighboring fully-filled ice shelf cells.  If option \texttt{-part_grid} is used then the PISM output file will have field \texttt{Href} which shows the amount of ice in the partially-filled cells as a thickness.  When a cell becomes fully-filled, in the sense that the \texttt{Href} thickness equals the average of neighbors, then the residual mass is redistributed to neighboring partially-filled or empty grid cells if option \intextoption{part_redist} is set.

The stress balance equations determining the velocities are only sensitive to ``fully-filled'' cells.  Similarly, advection is controlled by values of velocity in fully-filled cells.  Adaptive time stepping (specifically: the CFL criterion) limits the speed of ice front propagation so that at most one empty cell is filled, or one full cell emptied, per time step by the advance or retreat, respectively, of the calving front.

\begin{table}[ht]
  \centering
 \begin{tabular}{lp{0.7\linewidth}}
    \\\toprule
    \textbf{Option} & \textbf{Description}
    \\\midrule
    \intextoption{cfbc} & apply the stress boundary condition along the ice shelf calving front\\
    \intextoption{kill_icebergs} & identify and eliminate free-floating icebergs, which cause well-posedness problems for the SSA stress balance solver \\
    \intextoption{part_grid} & allow the ice shelf front to advance by a part of a grid cell, avoiding
	the development of unphysically-thinned ice shelves\\
    \intextoption{part_redist} &  scheme which makes the -part_grid mechanism conserve mass\\ 
    \midrule
    \intextoption{pik} & equivalent to option combination ``\texttt{-cfbc -kill_icebergs -part_grid -part_redist}'' \\
    \bottomrule
 \end{tabular}
\caption{Options which turn on PISM-PIK ice shelf front mechanisms.}
\label{tab:pism-pik-part-grid}
\end{table}


\subsection{Stress condition at calving fronts}
\label{sec:cfbc}
The vertically integrated force balance at floating calving fronts has been formulated by \cite{Morland} as
\begin{equation}
\int_{z_s-\frac{\rho}{\rho_w}H}^{z_s+(1-\frac{\rho}{\rho_w})H}\mathbf{\sigma}\cdot\mathbf{n}\;dz = \int_{z_s-\frac{\rho}{\rho_w}H}^{z_s}\rho_w g (z-z_s) \;\mathbf{n}\;dz.
\label{MacAyeal2}
\end{equation}
with $\mathbf{n}$ being the horizontal normal vector pointing from the ice boundary oceanward, $\mathbf{\sigma}$ the \emph{Cauchy} stress tensor, $H$ the ice thickness and $\rho$ and $\rho_{w}$ the densities of ice and seawater, respectively, for a sea level of $z_s$. The integration limits on the right hand side of Eq.~\eqref{MacAyeal2} account for the pressure exerted by the ocean on that part of the shelf, which is below sea level (bending and torque neglected). The limits on the left hand side change for water-terminating outlet glacier or glacier fronts above sea level according to the bed topography. Applying the ice flow law (Sect.~\ref{sec:rheology}) Eq.~\eqref{MacAyeal2} can be written in terms of strain rates (velocity derivatives).

Note that the discretized SSA stress balance, in the default finite difference discretization chosen by \intextoption{ssa_method} \texttt{fd}, is solved with an iterative matrix scheme. During matrix assembly, those grid cells along the ice domain boundary (fully-filled) are replaced according to Eq.~\eqref{MacAyeal2} to apply the correct forces, when option \intextoption{cfbc} is set.  Details can be found in \cite{Winkelmannetal2011} and \cite{Albrechtetal2011}.  


\subsection{Iceberg removal}
\label{sec:kill-icebergs}
Any calving mechanism (next subsection) removes ice along the seaward front of the ice shelf domain.  This can lead to isolated grid cells (filled or partially-filled) of floating ice, or to patches of floating ice (iceberg) fully surrounded by ice free ocean neighbors, and hence detached from the feeding ice sheet.  That is, calving can lead to icebergs.

In terms of our basic model of ice as a very-viscous fluid, however, the stress balance for an iceberg is not well-posed because the ocean applies no resistance to balance its driving stress \cite{SchoofStream}!  In this situation the numerical stress balance (SSA) solver will fail.

Option \intextoption{kill_iceberg} turns on the mechanism which cleans this up.  This option, or \intextoption{pik}, is generally needed if there is nontrivial calving.  It identifies such regions by checking iteratively for grid neighbors that are grounded, creating an ``iceberg mask'' showing floating ice that is, and is not (icebergs).  That is, it finds attachments back to grounded ice.  It then eliminates the free-floating icebergs.  This mass loss is reported as a scalar time series (see \ref{sec:saving-time-series}).


\subsection{Calving}
\label{sec:calving}
\optsection{Calving}
At the start, PISM-PIK included a physically based 2D-calving parameterization, further developed in \cite{Levermannetal2012}. This calving parameterization is turned on by option \intextoption{eigen_calving}.  Average calving rates, $c$, are proportional to the product of principal components of the horizontal strain rates, $\dot{\epsilon}_{_\pm}$, derived from SSA-velocities 
\begin{equation}
\label{eq: calv2}
c = K\; \dot{\epsilon}_{_+}\; \dot{\epsilon}_{_-}\quad\text{and}\quad\dot{\epsilon}_{_\pm}>0\:.
\end{equation}
The rate $c$ is in $\text{m}\,\text{s}^{-1}$, and the principal strain rates $\dot\eps_\pm$ have units $\text{s}^{-1}$, so $K$ has units $\text{m}\,\text{s}$.  The constant $K$ incorporating material properties of the ice at the front can be set using the \intextoption{eigen_calving_K} option or a configuration parameter (\texttt{eigen_calving_K} in \texttt{src/pism_config.cdl}).

The actual strain rate pattern strongly depends on the geometry and boundary conditions along the the confinements of an ice shelf (coast, ice rises, front position).  The strain rate pattern provides information in which regions preexisting fractures are likely to propagate, forming rifts (in two directions).  These rifts may ultimately intersect, leading to the release of icebergs. This and other ice shelf model calving models are not intended to resolve individual rifts or calving events. This first-order approach produces structurally-stable calving front positions which agree well with observations.  Calving rates balance terminal SSA velocities on average.

The partially-filled grid cell formulation (subsection \ref{sec:part-grid}) provides a framework suitable to relate the calving rate produced by \intextoption{eigen_calving} to the mass transport scheme at the ice shelf terminus.  Ice shelf front advance and retreat due to calving are limited to a maximum of one grid cell length per (adaptive) time step.

PISM also includes three more basic calving mechanisms (Table \ref{tab:calving}). The option \intextoption{thickness_calving} is based on the observation that ice shelf calving fronts are commonly thicker than about 150--250\,m (even though the physical reasons are not clear yet). Accordingly, any floating ice thinner than $H_{\textrm{cr}}$ is removed along the front, at a rate at most one grid cell per time step. The value of $H_{\mathrm{cr}}$ can be set using the \intextoption{calving_at_thickness} option or the \texttt{calving_at_thickness} configuration parameter.

Option \intextoption{float_kill} removes (calves), at each time step of the run, any ice that satisfies the flotation criterion.  Use of this option implies that there are no ice shelves in the model at all.

Option \fileopt{ocean_kill} reads in the ice thickness field \texttt{thk} from a file.  Any locations which were ice-free (\texttt{thk}$=0$) and which had bedrock elevation below sea level (\texttt{topg}$<0$), in the provided data set, are marked as ice-free ocean.  The resulting mask is not altered during the run, and is available as diagnostic field \texttt{ocean_kill_mask}.  At these places any floating ice is removed at each step of the run.  Ice shelves can exist in locations where a positive thickness was supplied in the provided data set.  If a file name is omitted then the ice thickness at the beginning of the run is used, that is, \texttt{thk} is read from the input file.

\begin{table}[ht]
  \centering
  \begin{tabular}{lp{0.6\linewidth}}
    \toprule
    \textbf{Option} & \textbf{Description} \\
    \midrule
    \intextoption{eigen_calving} & Physically-based calving parameterization \cite{Levermannetal2012,Winkelmannetal2011}.  Where ever the product of principle strain rates is positive, the calving rate is proportional to this product.  \\
    \intextoption{eigen_calving_K} ($K$) & Sets the proportionality parameter $K$ in $\text{m}\,\text{s}$. \\
    \intextoption{thickness_calving} & Calve all near-terminus ice which is thinner than ice threshold thickness $H_{\textrm{cr}}$. \\
    \intextoption{calving_at_thickness ($H_{\textrm{cr}}$)} & Sets the thickness threshold in $\text{m}$. \\
    \intextoption{float_kill} & All floating ice is calved off immediately.\\
    \fileopt{ocean_kill} & All ice flowing into grid cells marked as ``ice free ocean'', according to the ice thickness in the provided file, is calved. \\
    \bottomrule
  \end{tabular}
\caption{Calving models}
\label{tab:calving}
\end{table}


%%% Local Variables:
%%% mode: latex
%%% TeX-master: "manual"
%%% End:

% LocalWords:  html PISM PISM's
