
\section{Making \emph{HARD} Modeling Choices}
\label{sec:hard-choices}

Most uses of an ice sheet model depend on careful modeling choices in situations where there are considerable uncertainties \emph{and} the model results depend strongly on those choices.  There may be, at the present state of knowledge, \emph{no clear default values} that PISM can provide.

The current section collects some modeling choices which have this ``hard'' flavor.  The available PISM options related to these choices are known to be \emph{not} sufficient for all users.  These are modeling choices for which \emph{we know} the user may have to do a great deal more hard work than just choose among PISM runtime options.  Here are real example cases where users have worked hard:
\begin{itemize}
\item User was required to make use of available data in order to choose parameters for existing PISM models.  These parameters will then override PISM defaults:
\begin{center} % our UAF current situation with Greenland
\fbox{ \begin{minipage}[t]{5.0in}
\emph{Example}.  Use regional atmosphere model output to identify PDD parameters suitable for modeling surface mass balance on a particular ice sheet.  Then supply these parameters to PISM by a \texttt{-config\_override} file.
\end{minipage} }
\end{center}
\item User was required to write code, including code which modified current PISM internals, either to add additional processes or to ``correct'' PISM default process models; or 
\begin{center} % e.g. harder case is Potsdam marine ice sheet mods
\fbox{ \begin{minipage}[t]{5.0in}
\emph{Example}.  Add a new sub-ice-shelf melt model by modifying C++ code in the \texttt{src/coupler/} directory.
\end{minipage} }
\end{center}
\item User deliberately simplified the model in use, instead of the default which was ``throwing in the kitchen sink''.
\begin{center} % Nick's -hold_tauc choice
\fbox{ \begin{minipage}[t]{5.0in}
\emph{Example}.  Instead of using the PISM default mechanism connecting basal melt rate and basal strength, bypass this mechanism and impose a map of yield stress \texttt{tauc} directly.
\end{minipage} }
\end{center}
\end{itemize}

Obviously there is no actual clear distinction between the ``hard'' choices in this section and the ``easy'' ones in the previous section, but the subsections here cover issues for which the PISM developers have a hard time supplying easy answers!

\subsection{Controlling basal strength}  \label{subsect:basestrength}
\optsection{Basal strength and sliding}

In the \texttt{-ssa_sliding} case of a SIA+SSA hybrid model, the determination of basal resistance comes in part from a stored basal till material property $\phi=$\texttt{tillphi}, the till friction angle \cite{Paterson}.  The actual strength value is a till yield stress $\tau_c$, which necessarily represents the strength of the aggregate material at the base of an ice sheet, a poorly observed mixture of liquid water, ice, and granular till.  The value of $\tau_c$ is determined in part by a basal water model and in part by $\phi$, using a Mohr-Coulomb criterion \cite[Chapter 8]{Paterson}, 
\begin{equation*}
   \tau_c = c_{0} + (\tan\phi)\,(\rho g H - p_w).
\end{equation*}
Here $H$ is the ice thickness, $\rho$ the ice density, $g$ the acceleration of gravity, $p_w$ is the modeled pore water pressure, and $\phi$ is the till friction angle.  The difference $(\rho g H - p_w)$ is the modeled value of the effective pressure on the material till.  Note Schoof \cite{SchoofStream} formula (2.4) uses till cohesion $c_0 = 0$ and that is the default value in PISM.

\begin{table}
  \centering
 \begin{tabular}{lp{0.6\linewidth}}
    \\\toprule
    \textbf{Option} & \textbf{Description}
    \\\midrule
    \txtopt{topg_to_phi}{\emph{list of 4 numbers}} & Compute $\phi$ using \eqref{eq:1} and \eqref{eq:2}.\\
    \intextoption{pseudo_plastic} & enables the pseudo-plastic till model \\
    \txtopt{plastic_pwfrac}{\emph{pure number}} & Set what fraction of overburden pressure is assumed as the till pore water pressure.  Only relevant at basal points where there is a positive amount of basal water.\\
    \intextoption{plastic_c0} & Set the value of the till cohesion ($c_{0}$) in the plastic till model.  The value is a pressure, given in kPa.\\
    \txtopt{plastic_reg}{(m/a)} & Set the value of $\eps$ regularization of plastic till; this is the second ``$\eps$'' in formula (4.1) in \cite{SchoofStream}. The default is $0.01$.\\
    \txtopt{plastic_phi}{(degrees)} & Use a constant till friction angle. The default is $30^{\circ}$.\\
    \intextoption{pseudo_plastic_q} & Set the exponent $q$.\\
    \txtopt{pseudo_plastic_uthreshold}{(m/a)} & Set $u_{\text{threshold}}$. The default is $100$ m/a.\\
    \intextoption{hold_tauc} &   Keep the current values of the till yield stress $\tau_c$.  That is, do not update them by the default model using the stored basal melt water.  Only effective if \texttt{-ssa_sliding} is also set.
   \\\bottomrule
  \end{tabular}
\caption{Basal strength command-line options}
\label{tab:basal-strength}
\end{table}

Option \texttt{-plastic_pwfrac} determines $\alpha$, the quantity controlling how $p_w$ is determined from the effective thickness of basal water, the quantity $w=\mathtt{bwat}$; see the next subsection.  The formula is $p_w = \alpha\, w \rho g H$.  See \cite{BKAJS}.

We find that an effective, though heuristic, way to determine \texttt{tillphi} is to make it a function of bed elevation \cite{BKAJS}.  This heuristic is motivated by hypothesis that basal material with a marine history should be weak \cite{HuybrechtsdeWolde}.

Thus PISM has a mechanism setting $\phi$=\texttt{tillphi} to a \emph{piecewise-linear} function of bed elevation.  Given 4 parameters: $\phi_{\mathrm{min}}$, $\phi_{\mathrm{max}}$, $b_{\mathrm{min}}$, $b_{\mathrm{max}}$, where $b$ stands for the bed elevation, let 
\begin{equation}
  M = (\phi_{\text{max}} - \phi_{\text{min}}) / (b_{\text{max}} - b_{\text{min}})\label{eq:1}
\end{equation}
be the slope of the nontrivial part.  Then
\begin{equation}
  \phi(x,y) = \begin{cases}
    \phi_{\text{min}}, & b(x,y) \le b_{\text{min}}, \\
    \phi_{\text{min}} + (b(x,y) - b_{\text{min}}) \,M,
    &  b_{\text{min}} < b(x,y) < b_{\text{max}}, \\
    \phi_{\text{max}}, & b_{\text{max}} \le b(x,y). \end{cases}\label{eq:2}
\end{equation}

Actually the ``yield stress'' $\tau_c$ can be part of a power law model.  In fact, the basal resistance is normally determined by a user-configurable ``pseudo-plastic'' power law.  Specifically, stress is generally a power law function of basal sliding velocity $\mathbf{u}$:
   $$\tau_b = \tau_c \frac{|\mathbf{u}|^{q-1}}{u_{\text{threshold}}^q}\, \mathbf{u}.$$
Here $\tau_c$ corresponds to the variable \texttt{tauc} in PISM input and output files, $q$ is the power controlled by \texttt{-pseudo_plastic_q}, and the threshold velocity $u_{\text{threshold}}$ is controlled by \texttt{-pseudo_plastic_uthreshold}.  The purely plastic case is $q=0$---see \cite{SchoofStream} for precise interpretation---and the linear case is $q=1$, in which case the coefficient of velocity, $\tau_c/u_{\text{threshold}}$, is more commonly called $\beta$ or $\beta^2$ \cite{MacAyeal}.

\begin{quote}
  \textbf{WARNING!} Options \texttt{-pseudo_plastic_q} and \texttt{-pseudo_plastic_uthreshold} have no effect if \texttt{-preudo_plastic} is not set.
\end{quote}


See \emph{PISM Source Code browser}, source files in \texttt{src/base/basal_resistance}, and \cite{BBssasliding,BKAJS} for more details.

The major example of \texttt{-ssa_sliding} usage is in the first section of this manual.  A made-up example, in which sliding happens in the ``trough'', is
\begin{verbatim}
pisms -eisII I -ssa_sliding -Mx 91 -My 91 -Mz 51 \
      -topg_to_phi 5.0,15.0,0.0,1000.0 -y 12000
\end{verbatim}

A final note on basal sliding is in order.  There can be sliding in the SIA stress balance model (paradigm) itself, where the velocity of sliding is a direct function of the driving stress.  Such a SIA sliding mechanism appears in ISMIP-HEINO \cite{Calovetal2009HEINOfinal} and many other SIA-based modeling efforts in the literature.  \emph{This kind of sliding is not recommended, as it does not make sense to regard the driving stress as the local generator of flow if the bed is not holding all of that stress.}  Within PISM there is an implementation of SIA-based sliding for the verification test E, see \texttt{SIA_Sliding.cc}.  PISM does \emph{not} support this sliding mode in other contexts.


\subsection{Subglacial hydrology}  \label{subsect:subhydro}
\optsection{Subglacial hydrology}

Currently, only the most minimal possible hydrology model is implemented in PISM.  By default, the energy conservation calculation generates basal melt and that water is stored locally in the ``till'' under the ice sheet.  The output variable \texttt{bwat} is the effective thickness of this layer of liquid water.  This layer of water relates to the basal boundary condition of the conservation of energy scheme, and it is involved in computing basal water pressure and thus the till yield stress; see the previous subsection.

The minimal model which is on by default is that water is added by basal melt rate, subtracted by refreeze onto the base of the ice, and it decays away in the absence of other inputs according to the configuration parameter \texttt{bwat_decay_rate}.  The amount is bounded above by the configuration constant \texttt{bwat_max}, an effective thickness.  Water above that level is lost in an unmodeled manner.

There is also the option of horizontal diffusion of stored basal water in the field \texttt{bwat}.  This mechanism is normally off, but it is turned on by the option \intextoption{diffuse_bwat}.  The configuration parameters \texttt{bwat_diffusion_distance} and \texttt{bwat_diffusion_time} control the diffusivity.  See equation (11) in \cite{BBssasliding}.

% undocumented options/config. parameters
% base/iMoptions.cc:  "thk_eff" : "thk_eff_basal_water_pressure"
% base/iMoptions.cc:  "thk_eff_H_high" : "thk_eff_H_high"
% base/iMoptions.cc:  "thk_eff_H_low" : "thk_eff_H_low"
% base/iMoptions.cc:  "thk_eff_reduced" : "thk_eff_reduced"
% base/iMoptions.cc:  "bmr_enhance" : "bmr_enhance_basal_water_pressure"
% base/iMoptions.cc:  "bmr_enhance_scale" : "bmr_enhance_scale"


%%% Local Variables: 
%%% mode: latex
%%% TeX-master: "manual"
%%% End: 

% LocalWords:  PDD pdd html PISM PISM's paleo
