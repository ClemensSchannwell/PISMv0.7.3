% This file contains macros included in manual.tex and forcing.tex, in the preamble.
% This way we have all the URLs and such in one place.

%% THE FOLLOWING SHOULD CHANGE FOR A STABLE RELEASE:
\newcommand{\PISMREV}{revision \texttt{@Pism_REVISION_TAG@}}
\newcommand{\PETSCREL}{3.2}
\newcommand{\PISMDOWNLOADMSG}{Get development branch source code:
  \quad \texttt{git clone -b dev git@github.com:pism/pism.git pism-dev} \quad}
\newcommand{\PISMBROWSERURL}{http://www.pism-docs.org/wiki/doku.php?id=browser}
\newcommand{\PISMEMAIL}{\href{mailto:help@pism-docs.org}{help@pism-docs.org}}

\newcommand{\normalspacing}{\renewcommand{\baselinestretch}{1.1}\tiny\normalsize}
\newcommand{\tablespacing}{\renewcommand{\baselinestretch}{1.0}\tiny\normalsize}
\normalspacing

\usepackage[usenames]{xcolor}

\usepackage{bm,url,xspace,verbatim}
\usepackage{amssymb,amsmath}
\usepackage[pdftex]{graphicx}

\usepackage{booktabs}           % better rules in tables
\usepackage{xtab}               % long (multi-page) tables
\usepackage[nohyphen]{underscore}

\newcommand{\ddt}[1]{\ensuremath{\frac{\partial #1}{\partial t}}}
\newcommand{\ddx}[1]{\ensuremath{\frac{\partial #1}{\partial x}}}
\newcommand{\ddy}[1]{\ensuremath{\frac{\partial #1}{\partial y}}}
\renewcommand{\t}[1]{\texttt{#1}}
\newcommand{\Matlab}{\textsc{Matlab}\xspace}
\newcommand{\bq}{\mathbf{q}}
\newcommand{\bU}{\mathbf{U}}
\newcommand{\eps}{\epsilon}
\newcommand{\grad}{\nabla}
\newcommand{\Div}{\nabla\cdot}

%% macros having to do with documentation for options; note these appear in the index

\newindex{default}{idx}{ind}{General Index}
\newindex{options}{odx}{ond}{PISM Command-line options}

\def\optsection#1{%
  \def\optindex##1{\index[options]{#1!##1}}
  \def\optseealso##1{\index[options]{#1|see{##1}}}
}

\optsection{FIXME}

% Use this to index option definitions:
\newcommand{\intextoption}[1]{\texttt{-#1}\optindex{\texttt{-#1}}}

\newcommand{\txtopt}[2]{\texttt{-#1} #2\optindex{\texttt{-#1} #2}}

\newcommand{\listopt}[1]{\txtopt{#1}{\emph{comma-separated list}}}
\newcommand{\fileopt}[1]{\txtopt{#1}{\emph{filename}}}
\newcommand{\timeopt}[1]{\txtopt{#1}{\emph{range or list}}}
