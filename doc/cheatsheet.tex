\documentclass[landscape]{article}
\usepackage{scalefnt}
\usepackage{underscore}

\usepackage{graphicx}
\usepackage[pdftex]{hyperref}
\hypersetup{pdftitle={PISM Cheat Sheet}}
\hypersetup{pdfauthor={PISM (help@pism-docs.org)}}
\hypersetup{pdfkeywords={cheat sheet, ice sheet model, parallel ice
    sheet model, }}
\hypersetup{pdfsubject={get PISM sources at http://www.pism-docs.org}}
\hypersetup{colorlinks,
  plainpages=false, % only if colorlinks=true
  linkcolor=blue,   % only if colorlinks=true
  citecolor=black,   % only if colorlinks=true
  urlcolor=blue     % only if colorlinks=true
}

% The textpos package is necessary to position textblocks at arbitrary 
% places on the page.
%\usepackage[absolute,showboxes]{textpos}
\usepackage[absolute]{textpos}

\TPGrid[10mm,5mm]{47}{20}      % 3 cols of width 15, plus 2 gaps width 1

\newcommand{\PISMREV}{trunk revision \input{revision.tex}}
\newcommand{\PETSCREL}{2.3.3 or 3.0.0}
\newcommand{\PISMDOWNLOADMSG}{Get development version of PISM source code: \texttt{svn co http://svn.gna.org/svn/pism/trunk pism-dev}}

% Don't print section numbers
\setcounter{secnumdepth}{0}

\parindent=0pt
\parskip=0.5\baselineskip

\sloppy

\newcommand{\tabletitle}[1]{\multicolumn{2}{c}{\textbf{#1}}}

\begin{document}
\scalefont{0.8}
\pagestyle{empty}

%%%%% Header
\begin{textblock}{47}(0,0)
  \begin{center}
    \Large{\textbf{PISM (a Parallel Ice Sheet Model) Cheat Sheet} based on \PISMREV}
  \end{center}
\end{textblock}
%%%%%

\begin{textblock}{15}(0,0.8)
\section{Input and output}
\label{sec:input-output}

\begin{tabular}{@{}p{0.2\linewidth}p{0.75\linewidth}@{}}
\tabletitle{File I/O} \\
\texttt{-i} & selects an input file for restarting \\
\texttt{-boot_from} & selects a file to bootstrap from \\
\texttt{-o} & selects an output file \\
\texttt{-o_size} & specifies the ``size'' of the output; \texttt{small} means
``only model state variables'', \texttt{medium} adds some
diagnostic quantities, \texttt{big} saves \emph{every} supported diagnostic quantity\\
\tabletitle{Standard output} \\
\texttt{-verbose} & selects standard output verbosity level; \mbox{0 -- silent}; \mbox{1 --
errors only}; \mbox{2 -- the default}; 3, 4, 5 provide more info\\
\texttt{-help} & prints PISM and PETSc command-line option help; use with \texttt{grep}\\
\end{tabular}

\section{Domain and grid}
\label{sec:grid-setup}
  
\begin{tabular}{@{}p{0.2\linewidth}p{0.75\linewidth}@{}}
\tabletitle{Domain dimensions} \\
 \texttt{-Lx} (km) & half-width of the domain \mbox{(x-direction)}\\
  \texttt{-Ly} (km) & half-width of the domain \mbox{(y-direction)}\\
  \texttt{-Lz} (m) & domain height (in the ice)\\
  \texttt{-Lbz} (m) & domain height (in the bedrock)\\
\tabletitle{Grid size} \\
 \texttt{-Mx} & number of points in the x-direction\\
  \texttt{-My} & number of points in the x-direction\\
  \texttt{-Mz} & number of points in the z-direction in the ice\\
  \texttt{-Mbz} & number of points in the z-direction in the bedrock\\
\tabletitle{Vertical grid spacing} \\
 \texttt{-z_spacing} & vertical spacing in the ice; choose one of \texttt{quadratic} (default), \texttt{equal}\\
  \texttt{-zb_spacing} & same as \texttt{-z_spacing}, in the bedrock\\
\end{tabular}

\section{Model time}
\label{sec:model-time}
\begin{tabular}{@{}p{0.2\linewidth}p{0.75\linewidth}@{}}
  \texttt{-ys} & start year \\
  \texttt{-ye} & end year \\
  \texttt{-y} & run length
\end{tabular}

\section{Time-stepping}
\label{sec:time-stepping}
\begin{tabular}{@{}p{0.2\linewidth}p{0.75\linewidth}@{}}
  \texttt{-max_dt} & maximum time-step, in years; time steps determined by
  the adaptive mechanism will not exceed this\\
  \texttt{-skip} &  sets the maximum number of mass-balance steps to perform
  before the temperature, age, and SSA stress balance computations
  are done\\
\end{tabular}



\end{textblock}

\begin{textblock}{15}(16,0.8)


\section{Stress balance}
\label{sec:stress-balance}

\begin{tabular}{@{}p{0.35\linewidth}p{0.6\linewidth}@{}}
\texttt{-ssa_sliding} & use SSA as a sliding law\\
\texttt{-ssa_floating_only} & use SSA for floating ice only\\
\texttt{-eta} & use $\eta$ transformation in the surface gradient computation\\
\end{tabular}

\section{Rheology}
\label{sec:rheology}
\begin{tabular}{@{}p{0.35\linewidth}p{0.6\linewidth}@{}}
\texttt{-e} & enhancement factor\\
\texttt{-ice_type} & select a flow law; the default (polythermal) mode only
supports \texttt{gpbld}, \texttt{-cold} mode allows selecting from \texttt{pb},
\texttt{custom}, \texttt{hooke}, \texttt{arr}, \texttt{arrwarm},
\texttt{hybrid}, \texttt{gpbld}\\
\end{tabular}


\section{Basal strength and sliding}
\label{sec:basal-strength}
\begin{tabular}{@{}p{0.35\linewidth}p{0.6\linewidth}@{}}
\texttt{-hold_tauc} & keep the current values of the till yield stress\\
\texttt{-plastic_c0} & set the value of the till cohesion ($c_{0}$) in the
plastic till model\\
\texttt{-plastic_phi} & set the till friction angle (in degrees)\\
\texttt{-plastic_pwfrac} & set the maximum pore water fraction\\
\texttt{-pseudo_plastic_q} & set the exponent $q$ in the pseudo-plastic basal
sliding law\\
\texttt{-plastic_reg} (m/a) & set the value of $\epsilon$ regularization of plastic
till\\
\texttt{-pseudo_plastic_uthreshold} & set $u_{\mathrm{threshold}}$\\
\texttt{-topg_to_phi} & compute till friction angle as a function of bed
elevation; arguments are $\phi_{\mathrm{min}}$, $\phi_{\mathrm{max}}$, $b_{\mathrm{min}}$, $b_{\mathrm{max}}$, $\phi_{\mathrm{ocean}}$
\end{tabular}


\section{Age of the ice}
\label{sec:age}
\begin{tabular}{@{}p{0.35\linewidth}p{0.6\linewidth}@{}}
\texttt{-age} & compute age of the ice; \mbox{default: off}\\
\end{tabular}

\section{Bed deformation}
\label{sec:bed-deformation}
\begin{tabular}{@{}p{0.35\linewidth}p{0.6\linewidth}@{}}
  \texttt{-bed_def} & turns on a bed deformation model; choose one of \texttt{iso} (isostatic), \texttt{lc}~(Lingle-Clark);\\
\end{tabular}



\end{textblock}

\begin{textblock}{15}(32,0.8)


\section{Atmosphere models}
\label{sec:atmosphere-models}

\begin{tabular}{@{}p{0.3\linewidth}p{0.65\linewidth}@{}}
\texttt{-atmosphere} & selects an atmosphere model; choose one of
\mbox{\texttt{searise_greenland}}, \texttt{constant},
\mbox{\texttt{temp_lapse_rate}}\\
\tabletitle{\texttt{searise_greenland} options}\\
\texttt{-paleo_precip} & turn on SeaRISE-Greenland paleo-precipitation
correction; use with \texttt{-dTforcing}\\
\tabletitle{\texttt{temp_lapse_rate} options}\\
\texttt{-lapse_rate} & set the lapse rate (in $^{\circ}$C/m)\\
\tabletitle{Forcing} \\
\texttt{-dTforcing} & mean-annual air temperature forcing (ice-core-based,
scalar)\\
\texttt{-anomaly_precip} & precipitation forcing\\
\texttt{-anomaly_temp_ma} & mean-annual air temperature forcing\\
\end{tabular}

\section{Surface (snow) processes models}
\label{sec:surface-models}

\begin{tabular}{@{}p{0.3\linewidth}p{0.65\linewidth}@{}}
  \texttt{-surface} & selects a surface (snow) processes model; choose one of
  \texttt{simple}, \texttt{constant}, \texttt{pdd}\\
  \tabletitle{\texttt{pdd} options} \\
  \texttt{-pdd_fausto} & use Fausto et al (2009) Greenland PDD parameters\\
  \texttt{-pdd_rand} & use a monte carlo simulation to simulate daily temperature variation\\
 \texttt{-pdd_rand_repeatable} & use a repeatable monte carlo simulation to simulate daily temperature variation\\
  \texttt{-pdd_std_dev} & set PDD standard deviation\\
  \tabletitle{Forcing} \\
  \texttt{-force_to_thk} & select a file with the target ice thickness
  distribution\\
  \texttt{-force_to_thk_alpha} & set $\alpha$
\end{tabular}

\section{Ocean models}
\label{sec:ocean-models}

\begin{tabular}{@{}p{0.3\linewidth}p{0.65\linewidth}@{}}
\texttt{-ocean} & selects an ocean model; the only (default) choice is
\texttt{constant} \\
\tabletitle{Forcing} \\
\texttt{-dSLforcing} & sea level elevation forcing\\
\end{tabular}

\section{Calving}
\label{sec:calving}
\begin{tabular}{@{}p{0.3\linewidth}p{0.65\linewidth}@{}}
  \texttt{-float_kill} & calve off floating ice (calving at the grounding
  line)\\
\texttt{-ocean_kill} & calve off the ice entering cells marked as ``ice-free
ocean at the beginning of the run''
\end{tabular}

\end{textblock}

\null\newpage
%%%%% Footer
\begin{textblock}{47}(0,18)
  \begin{center}
    \hrulefill\\
    This cheat-sheet may be freely distributed under the terms of the GNU general
    public licence --- Copyright \copyright\ 2010 by PISM --- v0.3 ---
    \href{http://www.pism-docs.org}{\texttt{www.pism-docs.org}}\\
    \PISMDOWNLOADMSG\\
    To print this cheat-sheet, set your printer to ``two-sided, short-edge binding''.
 \end{center}
\end{textblock}
%%%%%

\begin{textblock}{15}(0,1)

\section{Disabling PISM components}
\label{sec:switches}
\begin{tabular}{@{}p{0.3\linewidth}p{0.65\linewidth}@{}}
\texttt{-cold} & disable polythermal code and use temperature in the energy
balance computation\\
\texttt{-no_mass} & disable mass continuity computations; \mbox{default: on}\\
\texttt{-no_temp} & disable energy balance computations: \mbox{default: on}
\end{tabular}

\section{PISM configuration file}
\label{sec:pism-config-file}
\begin{tabular}{@{}p{0.3\linewidth}p{0.65\linewidth}@{}}
\texttt{-config} & read model flags and parameters from an alternative config file\\
\texttt{-config_override} & select a file to use for overriding config
settings; useful for parameter studies\\
\texttt{-dump_config} & save configuration flags and parameters used in a run to a file
\end{tabular}

\section{Regridding}
\label{sec:regridding}
\begin{tabular}{@{}p{0.3\linewidth}p{0.65\linewidth}@{}}
  \texttt{-regrid_file} & select a file to regrid from\\
  \texttt{-regrid_vars} & a list of variables to regrid; choose from
  \texttt{age}, \texttt{bwat}, \texttt{dbdt}, \texttt{enthalpy}, \texttt{litho_temp}, \texttt{temp},
  \texttt{thk}, \texttt{topg}\\
\end{tabular}

\section{Saving model snapshots}
\label{sec:snapshots}
\begin{tabular}{@{}p{0.3\linewidth}p{0.65\linewidth}@{}}
\texttt{-save_file} & set the file save to\\
\texttt{-save_times} & a range or list of times\\
\texttt{-split_snapshots} & save snapshots to separate files\\
\texttt{-save_size} & adjust the ``size'' of a snapshot; select from
\texttt{small}, \texttt{medium}, \texttt{big}. Compare to \texttt{-o_size}.
\end{tabular}

\section{Saving 2D and 3D time-series}
\label{sec:extras}
\begin{tabular}{@{}p{0.3\linewidth}p{0.65\linewidth}@{}}
\texttt{-extra_file} & Specifies the file to save to.\\
\texttt{-extra_times} & Specifies times to save at, similar to
\texttt{-save_times} and \mbox{\texttt{-ts_times}}. \\
\texttt{-extra_vars} & select (list) variables to save\\
\texttt{-extra_split} & save time-series to separate files\\
\texttt{-extra_append} & append to an existing file\\
\end{tabular}

\end{textblock}

\begin{textblock}{15}(16,1)

\section{Saving scalar time-series}
\begin{tabular}{@{}p{0.35\linewidth}p{0.6\linewidth}@{}}
  \texttt{-ts_file} & Specifies the file to save to.\\
  \texttt{-ts_times} & Specifies times to save at either as a MATLAB-style range $a:\Delta t:b$ or a comma-separated list. \\
  \texttt{-ts_vars} & select (list) variables to save\\
  \texttt{-ts_append} & append to an existing file\\
\end{tabular}

\section{Scalar time-series}
\label{sec:scalar-time-series}

\begin{tabular}{@{}p{0.35\linewidth}p{0.6\linewidth}@{}}
\texttt{ivol} & total ice volume\\
\texttt{iarea} & total area covered by ice\\
\texttt{iareag} & total area covered by grounded ice\\
\texttt{iareaf} & total area covered by floating ice\\
\texttt{ienthalpy} & total ice enthalpy\\
\texttt{dt} & mass-continuity time step\\
% \texttt{divoldt} & rate of change of the total ice volume\\
% \texttt{dimassdt} & rate of change of the total ice mass\\
\end{tabular}

\section{Run-time viewers}
\label{sec:run-time-viewers}

\begin{tabular}{@{}p{0.35\linewidth}p{0.6\linewidth}@{}}
\texttt{-view_map} & a list of 2D fields to view\\
\texttt{-view_slice} & a list of 3D fields to view horizontal slices of\\
\texttt{-view_slice_level} & set the horizontal slice level, in meters above
the base of the ice\\
\texttt{-viewer_size} & set the viewer size \\
\texttt{-view_sounding} & a list of 3D fields to probe\\
\texttt{-id} & sounding row-index\\
\texttt{-jd} & sounding column-index\\
\texttt{-display} & Xwindows display; might be needed to use viewers with \texttt{mpiexec}\\
\texttt{-ksp_monitor_draw} & Iteration monitor for the Krylov subspace routines (KSP) 
in PETSc. Shows norm of residual versus iteration number.
\end{tabular}

\end{textblock}

\begin{textblock}{15}(32,1)
\section{Diagnostic quantities}
\label{sec:diagnostics}
\begin{tabular}{@{}p{0.2\linewidth}p{0.7\linewidth}@{}}
  \texttt{acab} &  ice-equivalent surface mass balance\\
  \texttt{age} &  age of ice \\
  \texttt{artm} &  annual average ice temperature at ice surface (below firn processes)\\
  \texttt{bfrict} &  basal frictional heating from ice sliding \\
  \texttt{bheatflx} &  geothermal flux at bedrock surface \\
  \texttt{bmelt} &  ice basal melt rate\\
  \texttt{bwat} &  thickness of subglacial melt water \\
  \texttt{bwp} &  subglacial (pore) water pressure \\
  \texttt{cbar} &  magnitude of vertically averaged horizontal ice velocity \\
  \texttt{cbase} &  magnitude of horizontal ice velocity at the base\\
  \texttt{cell_area} & cell areas computed using WGS84 datum\\
  \texttt{cflx} &  magnitude of vertically-integrated horizontal flux of ice \\
  \texttt{csurf} &  magnitude of horizontal ice velocity at ice surface \\
  \texttt{dHdt} &  rate of change of ice thickness \\
  \texttt{dbdt} &  bedrock uplift rate \\
  \texttt{dhdt} &  rate of change of surface elevation \\
  \texttt{enthalpybase} &  ice enthalpy at the ice base\\
  \texttt{enthalpysurf} &  ice enthalpy at the ice surface\\
  \texttt{hardav} &  vertical average of ice hardness \\
  \texttt{lat} &  latitude \\
  \texttt{litho_temp} &  lithosphere (bedrock) temperature\\
  \texttt{lon} &  longitude \\
  \texttt{mask} &  grounded/dragging/floating integer mask \\
  \texttt{tauc} &  yield stress for basal till \\
  \texttt{taud} &  magnitude of driving shear stress at base of ice \\
  \texttt{temp} &  ice temperature \\
  \texttt{temp_pa} &  pressure-adjusted ice temperature \\
  \texttt{tempbase} &  ice temperature at the ice base\\
  \texttt{temppabase} &  pressure-adjusted ice temperature at the base of ice\\
  \texttt{tempsurf} &  ice temperature at the ice surface\\
  \texttt{thk} &  ice thickness\\
  \texttt{tillphi} &  till friction angle \\
  \texttt{topg} &  bedrock surface elevation \\
  \texttt{usurf} &  ice upper surface elevation \\
  \texttt{\{u,v\}vel} &  horizontal ice velocity in the X,Y direction \\
  \texttt{uvbasal} &  horizontal  ice velocity at the ice base\\
  \texttt{\{u,v\}velsurf} &  horizontal ice velocity in the X,Y direction
  at the ice surface\\
  \texttt{uvbar} &  vertical mean of horizontal ice velocity\\
  \texttt{wvel} &  vertical ice velocity \\
  \texttt{wvelbase} &  vertical ice velocity at the ice base\\
  \texttt{wvelsurf} &  vertical ice velocity at the ice surface\\
  \end{tabular}

\end{textblock}

\end{document}

% LocalWords:  polythermal secnumdepth pdfauthor MATLAB pdftitle pdfkeywords Lx
% LocalWords:  pdfsubject colorlinks linkcolor citecolor urlcolor plainpages Lz
% LocalWords:  subglacial PETSc Lbz Mx Mz Mbz zb ys dt ssa gpbld pb hooke tauc
% LocalWords:  arrwarm pwfrac uthreshold topg iso lc searise greenland paleo et
% LocalWords:  precip SeaRISE dTforcing pdd fausto Fausto al monte carlo dev jd
% LocalWords:  thk dSLforcing config Regridding regrid bwat dbdt litho ivol ksp
% LocalWords:  iarea iareag iareaf divoldt dimassdt Xwindows mpiexec Krylov bwp
% LocalWords:  acab artm bfrict bheatflx bmelt cbar cbase WGS cflx csurf dHdt
% LocalWords:  dhdt enthalpybase enthalpysurf hardav lon taud tempbase tempsurf
% LocalWords:  tillphi usurf vel uvbasal velsurf uvbar wvel wvelbase wvelsurf
